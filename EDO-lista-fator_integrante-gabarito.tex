\everymath{\displaystyle}
%\documentclass[pdftex,a4paper]{article}
\documentclass[a4paper]{article}
%%classes: article, report, book, proc, amsproc

%%%%%%%%%%%%%%%%%%%%%%%%
%% Misc
% para acertar os acentos
\usepackage[brazilian]{babel} 
%\usepackage[portuguese]{babel} 
% \usepackage[english]{babel}
% \usepackage[T1]{fontenc}
% \usepackage[latin1]{inputenc}
\usepackage[utf8]{inputenc}
\usepackage{indentfirst}
\usepackage{fullpage}
% \usepackage{graphicx} %See PDF section
\usepackage{multicol}
\setlength{\columnseprule}{0.5pt}
\setlength{\columnsep}{20pt}
%%%%%%%%%%%%%%%%%%%%%%%%
%%%%%%%%%%%%%%%%%%%%%%%%
%% PDF support

\usepackage[pdftex]{color,graphicx}
% %% Hyper-refs
\usepackage[pdftex]{hyperref} % for printing
% \usepackage[pdftex,bookmarks,colorlinks]{hyperref} % for screen

%% \newif\ifPDF
%% \ifx\pdfoutput\undefined\PDFfalse
%% \else\ifnum\pdfoutput > 0\PDFtrue
%%      \else\PDFfalse
%%      \fi
%% \fi

%% \ifPDF
%%   \usepackage[T1]{fontenc}
%%   \usepackage{aeguill}
%%   \usepackage[pdftex]{graphicx,color}
%%   \usepackage[pdftex]{hyperref}
%% \else
%%   \usepackage[T1]{fontenc}
%%   \usepackage[dvips]{graphicx}
%%   \usepackage[dvips]{hyperref}
%% \fi

%%%%%%%%%%%%%%%%%%%%%%%%


%%%%%%%%%%%%%%%%%%%%%%%%
%% Math
\usepackage{amsmath,amsfonts,amssymb}
% para usar R de Real do jeito que o povo gosta
\usepackage{amsfonts} % \mathbb
% para usar as letras frescas como L de Espaco das Transf Lineares
% \usepackage{mathrsfs} % \mathscr

% Oferecer seno e tangente em pt, com os comandos usuais.
\providecommand{\sin}{} \renewcommand{\sin}{\hspace{2pt}\mathrm{sen}}
\providecommand{\tan}{} \renewcommand{\tan}{\hspace{2pt}\mathrm{tg}}

% dt of integrals = \ud t
\newcommand{\ud}{\mathrm{\ d}}
%%%%%%%%%%%%%%%%%%%%%%%%



\begin{document}

%%%%%%%%%%%%%%%%%%%%%%%%
%% Título e cabeçalho
%\noindent\parbox[c]{.15\textwidth}{\includegraphics[width=.15\textwidth]{logo}}\hfill
\parbox[c]{.825\textwidth}{\raggedright%
  \sffamily {\LARGE

Equações Diferenciais Ordinárias: Gabarito de Fatores Integrantes

\par\bigskip}
% {Centro Universitário Anhanguera de Niterói -- UNIAN\par} 
% {Curso: Engenharia\par}
{Prof: Felipe Figueiredo\par}
{\url{http://sites.google.com/site/proffelipefigueiredo}\par}
}

Versão: \verb|20141124|

%%%%%%%%%%%%%%%%%%%%%%%%


%%%%%%%%%%%%%%%%%%%%%%%%
%\section{}

% Equação Diferencial Ordinária Linear com coeficientes variáveis:

% \begin{displaymath}
%   y' + p(x)y = q(x)
% \end{displaymath}

% Fator integrante:
% \begin{displaymath}
%   \mu (x) = e^{\int p(x)\ud x}
% \end{displaymath}

% Família de soluções:
% \begin{displaymath}
%   y(x) = \frac{\int (\mu q) \ud x}{\mu}
% \end{displaymath}

%\section{}
\begin{enumerate}
\item % Encontre a família de soluções de cada uma das seguintes
  % equações diferenciais:

  \begin{enumerate}
  \item $y=\frac{e^x}{2}+ Ke^{-x}$ %$y' +y = e^x$
  \item $y=e^x(x+K)$ %$y' -y = e^x$
  \item $y=e^{2x} + Ke^x$ %$y' -y = e^{2x}$
  \item $y=-e^x+Ke^{2x}$ %$y' -2y = e^x$
  \item $y=x^5+K$ %$y' = 5x^4$
  \item $y=\frac{3}{4}x^2 +K$ %$2y' = 3x$
  \item $y=xe^x -1 +Ke^x = e^x(x+K) -1 $ %$y' -y = e^x + 1$
  \item $y=\frac{1}{5}e^{2x} -2 +Ke^{\frac{1}{2}x} $ %$2y' +y =e^{2x} - 2$
  \item $y=Ke^{2x} + \frac{1}{9}e^{-x} + \frac{1}{3}e^{x} +2$ %$-3y' + 6y -12 = e^{-x} + e^x$
  \item $y= Ke^{\frac{-2}{5}x} + \frac{5}{4}e^{2x} + \frac{10}{17}e^{3x} $ %$y' + \frac{2}{5}y = 3e^{2x} + 2e^{3x}$
  \item $y=K e^{-x} + x -1$ %$y' + y = x$
  \item $y=Ke^{(-x^2)} + \frac{1}{2} $ %$y' + 2xy = x$
  \item $y=Ke^{(-x^2)} + \frac{3}{2}$ %$\frac{1}{x}y' + 2y = 3$
  \item $y=K e^{(-x^3)} + \frac{1}{3} $ %$y' +3x^2y = x^2$
  \item $y=1+Ke^{(-\sin x)}$ %$y' + \cos (x) y = \cos x$
  \end{enumerate}

%\newpage
\item %Encontre a solução de cada PVI abaixo:

  \begin{enumerate}
  \item $y=\frac{e^x}{2}+ \frac{e^{-x}}{2} = \frac{e^x+ e^{-x}}{2} =
    \cosh x$ %$y' +y = e^x, y(0)=1$
  \item $y=xe^{x}$ %$y' -y = e^x,y(0)=0$
  \item $y=e^{2x}$ %$y' -y = e^{2x}, y(0)=1$
  \item $y=3e^{2x} - e^x$ %$y' -2y = e^x, y(0)=2$
  \item $y=x^5 +1$ %$y' = 5x^4, y(1)=2$
  \item $y=\frac{3}{4}x^2 -\frac{7}{4} = \frac{1}{4}(3x^2 -7)$ %$2y' = 3x, y(1)=-1$
  \item $y=xe^x-1$ %$y' -y = e^x + 1, y(0)=-1$
  \item $y=\frac{1}{5}e^{2x} -2 +\frac{15}{5}e^{\frac{1}{2}x}$ %$2y' +y =e^{2x} - 2, y(0)=1$
  \item $y=-\frac{22}{9}e^{2x} + \frac{1}{9}e^{-x} + \frac{1}{3}e^{x} +2$ %$-3y' + 6y -12 = e^{-x} + e^x, y(0)=0$
  \item $y=-\frac{57}{68}e^{\frac{-2}{5}x} + \frac{5}{4}e^{2x} + \frac{10}{17}e^{3x}$ %$y' + \frac{2}{5}y = 3e^{2x} + 2e^{3x}, y(0)=1$
  \item $y=2 e^{-x} + x -1$ %$y' + y = x, y(0)=1$
  \item $y=\frac{1}{2}(e^{(-x^2)} + 1)$ %$y' + 2xy = x, y(0)=1$
  \item $y=\frac{3}{2}-\frac{1}{2}e^{(-x^2)} $ %$\frac{1}{x}y' + 2y = 3, y(0)=1$
  \item $y= \frac{2}{3}e^{(-x^3)} + \frac{1}{3}$ %$y' +3x^2y = x^2, y(0)=1$
  \item $y=1-e^{-\sin x}$ %$y' + \cos (x) y = \cos x, y(0)=0$
  \end{enumerate}

\end{enumerate}


\end{document}
