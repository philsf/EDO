\everymath{\displaystyle}
%\documentclass[pdftex,a4paper]{article}
\documentclass[a4paper]{article}
%%classes: article, report, book, proc, amsproc

%%%%%%%%%%%%%%%%%%%%%%%%
%% Misc
% para acertar os acentos
\usepackage[brazilian]{babel} 
%\usepackage[portuguese]{babel} 
% \usepackage[english]{babel}
% \usepackage[T1]{fontenc}
% \usepackage[latin1]{inputenc}
\usepackage[utf8]{inputenc}
\usepackage{indentfirst}
\usepackage{fullpage}
% \usepackage{graphicx} %See PDF section
\usepackage{multicol}
\setlength{\columnseprule}{0.5pt}
\setlength{\columnsep}{20pt}
%%%%%%%%%%%%%%%%%%%%%%%%
%%%%%%%%%%%%%%%%%%%%%%%%
%% PDF support

\usepackage[pdftex]{color,graphicx}
% %% Hyper-refs
\usepackage[pdftex]{hyperref} % for printing
% \usepackage[pdftex,bookmarks,colorlinks]{hyperref} % for screen

%% \newif\ifPDF
%% \ifx\pdfoutput\undefined\PDFfalse
%% \else\ifnum\pdfoutput > 0\PDFtrue
%%      \else\PDFfalse
%%      \fi
%% \fi

%% \ifPDF
%%   \usepackage[T1]{fontenc}
%%   \usepackage{aeguill}
%%   \usepackage[pdftex]{graphicx,color}
%%   \usepackage[pdftex]{hyperref}
%% \else
%%   \usepackage[T1]{fontenc}
%%   \usepackage[dvips]{graphicx}
%%   \usepackage[dvips]{hyperref}
%% \fi

%%%%%%%%%%%%%%%%%%%%%%%%


%%%%%%%%%%%%%%%%%%%%%%%%
%% Math
\usepackage{amsmath,amsfonts,amssymb}
% para usar R de Real do jeito que o povo gosta
\usepackage{amsfonts} % \mathbb
% para usar as letras frescas como L de Espaco das Transf Lineares
% \usepackage{mathrsfs} % \mathscr

% Oferecer seno e tangente em pt, com os comandos usuais.
\providecommand{\sin}{} \renewcommand{\sin}{\hspace{2pt}\mathrm{sen}}
\providecommand{\tan}{} \renewcommand{\tan}{\hspace{2pt}\mathrm{tg}}

% dt of integrals = \ud t
\newcommand{\ud}{\mathrm{\ d}}
%%%%%%%%%%%%%%%%%%%%%%%%

\begin{document}

%%%%%%%%%%%%%%%%%%%%%%%%
%% Título e cabeçalho
%\noindent\parbox[c]{.15\textwidth}{\includegraphics[width=.15\textwidth]{logo}}\hfill
\parbox[c]{.825\textwidth}{\raggedright%
  \sffamily {\LARGE

Equações Diferenciais: Lista

EDOs de 1a ordem Separáveis

\par\bigskip}
% {Centro Universitário Anhanguera de Niterói -- UNIAN\par} 
% {Curso: Engenharia\par}
{Prof: Felipe Figueiredo\par}
{\url{http://sites.google.com/site/proffelipefigueiredo}\par}
}

Versão: \verb|20150724|

%%%%%%%%%%%%%%%%%%%%%%%%


%%%%%%%%%%%%%%%%%%%%%%%%

% \section{Formulário}

% Equação diferencial ordinária linear de primeira ordem com
% coeficientes constantes.

% \bigskip
% Equação:
% \begin{displaymath}
%   y' = ry
% \end{displaymath}

% Solução:
% \begin{displaymath}
%   y(x) = K e^{rx}
% \end{displaymath}

% onde $r \in \mathbb{R}, K \in \mathbb{R}$.

% \bigskip

% Equação:
% \begin{displaymath}
%   y' -ry = a
% \end{displaymath}

% Solução:
% \begin{displaymath}
%   y(x) = K e^{rx} - \frac{a}{r}
% \end{displaymath}

% onde $r \in \mathbb{R}^*, K \in \mathbb{R}, a \in \mathbb{R}$.

% \bigskip
% Obs: Evite usar as fórmulas acima em todos os exercícios. Faça o
% procedimento de resolução visto em sala tantas vezes quanto julgar
% necessário para adquirir a habilidade necessária.

\section{Exercícios}

\begin{enumerate}
\item Encontre as soluções das seguintes equações diferenciais lineares.

  \begin{enumerate}
  \item $y' -2y = 0$
  \item $y' -15y = 0$
  \item $y' = -3y$
  \item $y' = 9y$
  \item $y' - \frac{1}{2} y = 0$
  \item $y' = -\frac{1}{3} y$
  \item $2y' - 2y = 0$
  \item $5y' +5y =0 $
  \item $3y' -6y = 0 $
  \item $3y' -2y =0$
  \item $y' = \sqrt{3}y$
  \item $y' -2\pi y =0$
  \item $y' - ry = 0$
  \end{enumerate}


\item Encontre as soluções das seguintes equações diferenciais lineares.

  \begin{enumerate}
  \item $y' - 2y = 1$
  \item $y' - 10y = 2$
  \item $y' + y + 1 = 0$
  \item $y' + 1 = y$
  \item $2y' + 2y -1 = 0$
  \item $3y'- 4y +5 = 0$
  \item $\frac{1}{2} y' + \frac{1}{3} y + \frac{1}{4} = 0$
  \item $y' -ry = a$ (Sugestão: $ry+a = r(y+\frac{a}{r})$)
  \end{enumerate}

\item (Desafio) As seguintes equações diferenciais não-lineares podem
  ser resolvidas utilizando a mesma técnica dos exercícios
  anteriores. Encontre as soluções das equações abaixo.

  \begin{enumerate}
  \item $2y'y = 1$
  \item $y' = \frac{x}{y}$
  \item $y' = y^2$
  \end{enumerate}
\end{enumerate}

\end{document}
