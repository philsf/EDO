\everymath{\displaystyle}
%\documentclass[pdftex,a4paper]{article}
\documentclass[a4paper]{article}
%%classes: article, report, book, proc, amsproc

%%%%%%%%%%%%%%%%%%%%%%%%
%% Misc
% para acertar os acentos
\usepackage[brazilian]{babel} 
%\usepackage[portuguese]{babel} 
% \usepackage[english]{babel}
% \usepackage[T1]{fontenc}
% \usepackage[latin1]{inputenc}
\usepackage[utf8]{inputenc}
\usepackage{indentfirst}
\usepackage{fullpage}
% \usepackage{graphicx} %See PDF section
\usepackage{multicol}
\setlength{\columnseprule}{0.5pt}
\setlength{\columnsep}{20pt}
%%%%%%%%%%%%%%%%%%%%%%%%
%%%%%%%%%%%%%%%%%%%%%%%%
%% PDF support

\usepackage[pdftex]{color,graphicx}
% %% Hyper-refs
\usepackage[pdftex]{hyperref} % for printing
% \usepackage[pdftex,bookmarks,colorlinks]{hyperref} % for screen

%% \newif\ifPDF
%% \ifx\pdfoutput\undefined\PDFfalse
%% \else\ifnum\pdfoutput > 0\PDFtrue
%%      \else\PDFfalse
%%      \fi
%% \fi

%% \ifPDF
%%   \usepackage[T1]{fontenc}
%%   \usepackage{aeguill}
%%   \usepackage[pdftex]{graphicx,color}
%%   \usepackage[pdftex]{hyperref}
%% \else
%%   \usepackage[T1]{fontenc}
%%   \usepackage[dvips]{graphicx}
%%   \usepackage[dvips]{hyperref}
%% \fi

%%%%%%%%%%%%%%%%%%%%%%%%


%%%%%%%%%%%%%%%%%%%%%%%%
%% Math
\usepackage{amsmath,amsfonts,amssymb}
% para usar R de Real do jeito que o povo gosta
\usepackage{amsfonts} % \mathbb
% para usar as letras frescas como L de Espaco das Transf Lineares
% \usepackage{mathrsfs} % \mathscr

% Oferecer seno e tangente em pt, com os comandos usuais.
\providecommand{\sin}{} \renewcommand{\sin}{\hspace{2pt}\mathrm{sen}}
\providecommand{\tan}{} \renewcommand{\tan}{\hspace{2pt}\mathrm{tg}}

% dt of integrals = \ud t
\newcommand{\ud}{\mathrm{\ d}}
%%%%%%%%%%%%%%%%%%%%%%%%

\begin{document}

%%%%%%%%%%%%%%%%%%%%%%%%
%% Título e cabeçalho
%\noindent\parbox[c]{.15\textwidth}{\includegraphics[width=.15\textwidth]{logo}}\hfill
\parbox[c]{.825\textwidth}{\raggedright%
  \sffamily {\LARGE

Equações Diferenciais: Gabarito

Problemas de Valor Inicial e Campos de Direções

\par\bigskip}
% {Centro Universitário Anhanguera de Niterói -- UNIAN\par} 
% {Curso: Engenharia\par}
{Prof: Felipe Figueiredo\par}
{\url{http://sites.google.com/site/proffelipefigueiredo}\par}
}

Versão: \verb|20141124|

%%%%%%%%%%%%%%%%%%%%%%%%


%%%%%%%%%%%%%%%%%%%%%%%%

\section{Exercícios}

\begin{enumerate}
\item % Encontre a solução de cada um dos seguintes Problemas de Valor
  % Inicial:
  \begin{enumerate}
  \item $y = 4e^{2x}$ %$y' -2y = 0, y(0)=4$
  \item $y = 5e^{15x}$ %$y' -15y = 0, y(0)=5$
  \item $y = -2e^{-3x}$ %$y' = -3y, y(0)=-2$
  \item $y = -5e^{9x}$ %$y' = 9y, y(0)=-5$
  \item $y = -e^{\frac{1}{2}x}$ %$y' - \frac{1}{2} y = 0, y(0)=-1$
  \item $y = 10e^{-\frac{1}{3}x}$ %$y' = -\frac{1}{3} y, y(0)=10$
  \item $y = \frac{4}{3}e^{x}$ %$2y' - 2y = 0, y(0)=\frac{4}{3}$
  \item $y = -4e^{-x}$ %$5y' +5y =0, y(0)=-4$
  \item $y = -5e^{2x}$ %$3y' -6y = 0, y(0)=-5$
  \item $y = \pi e^{\frac{2}{3}x}$ %$3y' -2y =0, y(0)=\pi$
  \item $y = e^{\sqrt{3}x +2}$ %$y' = \sqrt{3}y, y(0)=e^2$
  \item $y = 5e^{2\pi x+1}$ %$y' -2\pi y =0, y(0)=5e$
  \item $y = y_0e^{rx}$ %$y' - ry = 0, y(0)=y_0$
  \end{enumerate}

\item % Encontre a solução de cada um dos seguintes Problemas de Valor
  % Inicial:
  \begin{enumerate}
  \item $y = e^{2x} - \frac{1}{2}$ %$y' - 2y = 1, y(0)=\frac{1}{2}$
  \item $y = e^{10x} - \frac{1}{5}$ %$y' - 10y = 2, y(0)=\frac{4}{5}$
  \item $y = -e^{x} + 1$ %$y' + y + 1 = 0, y(0)=0$
  \item $y = -3e^{x} + 1$ %$y' + 1 = y, y(0)=-2$
  \item $y = \frac{3}{2}e^{-x} + \frac{1}{2}$ %$2y' + 2y -1 = 0, y(0)=1$
  \item $y = -\frac{11}{12}e^{\frac{4}{3}x} + \frac{5}{4}$ %$3y'- 4y +5 = 0, y(0)=\frac{1}{3}$
  \item $y = -\frac{17}{4}e^{-\frac{2}{3}x} - \frac{3}{4}$ %$\frac{1}{2} y' + \frac{1}{3} y + \frac{1}{4} = 0, y(0)=-5$
  \item $y = (\frac{a}{r} + y_0)e^{rx} - \frac{a}{r}$ %$y' -ry = a, y(0)=y_0$
  \end{enumerate}

\item % Desenhe o Campo de Direções das seguintes equações, e esboce
  % algumas soluções das mesmas

  \begin{enumerate}
  \item %$y' = y-y^2$ (equação logística)
  \item %$y' = x$
  \item %$y' = \frac{x}{y}$
  \end{enumerate}
\end{enumerate}

\end{document}
