\everymath{\displaystyle}
%\documentclass[pdftex,a4paper]{article}
\documentclass[a4paper]{article}
%%classes: article, report, book, proc, amsproc

%%%%%%%%%%%%%%%%%%%%%%%%
%% Misc
% para acertar os acentos
\usepackage[brazilian]{babel} 
%\usepackage[portuguese]{babel} 
% \usepackage[english]{babel}
% \usepackage[T1]{fontenc}
% \usepackage[latin1]{inputenc}
\usepackage[utf8]{inputenc}
\usepackage{indentfirst}
\usepackage{fullpage}
% \usepackage{graphicx} %See PDF section
\usepackage{multicol}
\setlength{\columnseprule}{0.5pt}
\setlength{\columnsep}{20pt}
%%%%%%%%%%%%%%%%%%%%%%%%
%%%%%%%%%%%%%%%%%%%%%%%%
%% PDF support

\usepackage[pdftex]{color,graphicx}
% %% Hyper-refs
\usepackage[pdftex]{hyperref} % for printing
% \usepackage[pdftex,bookmarks,colorlinks]{hyperref} % for screen

%% \newif\ifPDF
%% \ifx\pdfoutput\undefined\PDFfalse
%% \else\ifnum\pdfoutput > 0\PDFtrue
%%      \else\PDFfalse
%%      \fi
%% \fi

%% \ifPDF
%%   \usepackage[T1]{fontenc}
%%   \usepackage{aeguill}
%%   \usepackage[pdftex]{graphicx,color}
%%   \usepackage[pdftex]{hyperref}
%% \else
%%   \usepackage[T1]{fontenc}
%%   \usepackage[dvips]{graphicx}
%%   \usepackage[dvips]{hyperref}
%% \fi

%%%%%%%%%%%%%%%%%%%%%%%%


%%%%%%%%%%%%%%%%%%%%%%%%
%% Math
\usepackage{amsmath,amsfonts,amssymb}
% para usar R de Real do jeito que o povo gosta
\usepackage{amsfonts} % \mathbb
% para usar as letras frescas como L de Espaco das Transf Lineares
% \usepackage{mathrsfs} % \mathscr

% Oferecer seno e tangente em pt, com os comandos usuais.
\providecommand{\sin}{} \renewcommand{\sin}{\hspace{2pt}\mathrm{sen}}
\providecommand{\tan}{} \renewcommand{\tan}{\hspace{2pt}\mathrm{tg}}

% dt of integrals = \ud t
\newcommand{\ud}{\mathrm{\ d}}
%%%%%%%%%%%%%%%%%%%%%%%%

\begin{document}

%%%%%%%%%%%%%%%%%%%%%%%%
%% Título e cabeçalho
%\noindent\parbox[c]{.15\textwidth}{\includegraphics[width=.15\textwidth]{logo}}\hfill
\parbox[c]{.825\textwidth}{\raggedright%
  \sffamily {\LARGE

Equações Diferenciais: Gabarito

Problemas de Valor Inicial e Campos de Direções

\par\bigskip}
% {Centro Universitário Anhanguera de Niterói -- UNIAN\par} 
% {Curso: Engenharia\par}
{Prof: Felipe Figueiredo\par}
{\url{http://sites.google.com/site/proffelipefigueiredo}\par}
}

Versão: \verb|20141124|

%%%%%%%%%%%%%%%%%%%%%%%%


%%%%%%%%%%%%%%%%%%%%%%%%

\section{Exercícios}

\begin{enumerate}
\item % Encontre a solução de cada um dos seguintes Problemas de Valor
  % Inicial:
  \begin{enumerate}
  % \item $y' -2y = 0$
  % \item $y' -15y = 0$
  % \item $y' = -3y$
  % \item $y' = 9y$
  % \item $y' - \frac{1}{2} y = 0$
  % \item $y' = -\frac{1}{3} y$
  % \item $2y' - 2y = 0$
  % \item $5y' +5y =0 $
  \item $y = -5e^{2x}$ %$3y' -6y = 0 , y(0) = -5$
  \item $y = \pi e^{\frac{2}{3}x}$ %$3y' -2y =0, y(0) = \pi$
  % \item (Desafio) $y' - ry = 0$
  \item $y = e^{2x} - \frac{1}{2}$ %$y' - 2y = 1, y(0) = \frac{1}{2}$
  \item $y = e^{10x} - \frac{1}{5}$ %$y' - 10y = 2, y(0) = \frac{4}{5}$
  % \item $y' + y + 1 = 0$
  % \item $y' + 1 = y$
  \item $y = \frac{e^{-x}}{2} + \frac{1}{2}$ %$2y' + 2y -1 = 0, y(0) = 1$
  \item $y = -\frac{11}{12}e^{\frac{4}{3}x} + \frac{5}{4}$ %$3y'- 4y +5 = 0, y(0) = \frac{1}{3}$
  \item $y = -\frac{17}{4}e^{-\frac{2}{3}x} - \frac{3}{4}$ %$\frac{1}{2} y' + \frac{1}{3} y + \frac{1}{4} = 0, y(0) = -5$
  \item $y = e^2e^{\sqrt{3}x}=e^{\sqrt{3}x+2}$ %$y' = \sqrt{3}y, y(0) = e^2$
  \item $y = (5e) e^{2\pi x} = 5e^{2\pi x +1}$ %$y' -2\pi y =0, y(0) = 5e$
  \item $y = 5x$ %$xy' = y, y(1) = 5$
  \item $y = e^{x^2}$ %$y' = 2xy, y(0)=1$
  \item $y = \pi e^{\sin x}$ %$y' = y\cos x, y(0)=\pi$
  \item $y = e^{x-1} x^x$ %$y' - y\ln x = 0, y(1)=1$
  \item $y = (\frac{a}{r} + y_0)e^{rx} - \frac{a}{r}$ %$y' -ry = a, y(0) = y_0$, onde $r$, $a$ e $y_0$ são constantes
  \end{enumerate}

\item % Encontre a solução de cada um dos seguintes Problemas de Valor
  % Inicial:
  \begin{enumerate}
  \item $y = \frac{1}{3-x}$ %$y' = y^2, y(1)=\frac{1}{2}$
  \item $y = \sqrt{x}$ %$2y'y = 1, y(1)=1$
  \item $y = \sqrt{2x-1}$ %$y' = \frac{1}{y}, y(1)=1$
  \item $y = -\sqrt{x^2+3}$ %$y' = \frac{x}{y}, y(1)=-2$
  \item $y = e^{\frac{x^3}{3}}$ %$y' = x^2y$
  \item $y = -\frac{2}{x^2+1}$ %$y' = xy^2, y(0)=-2$
  \item $y = -\frac{3}{x^3+2}$ %$y' = (xy)^2, y(1)=-1$
  \item $y = \sqrt{2\ln x +1}$ %$y' = \frac{1}{xy}, y(1)=1$
  \end{enumerate}

\item % Desenhe o Campo de Direções das seguintes equações, e esboce
  % algumas soluções das mesmas

  \begin{enumerate}
  \item %$y' = y-y^2$, soluções que começam em $y(0)=0$, $y(0)=\frac{1}{2}$ e $y(0)=1$
  \item %$y' = x$, soluções que começam em $y(0)=0$ e $y(0)=1$
  \item %$y' = \frac{x}{y}$, soluções que começam em $y(1)=-1$ e $y(1)=-2$
  \end{enumerate}
\end{enumerate}

\end{document}
