\everymath{\displaystyle}
%\documentclass[pdftex,a4paper]{article}
\documentclass[a4paper]{article}
%%classes: article, report, book, proc, amsproc

%%%%%%%%%%%%%%%%%%%%%%%%
%% Misc
% para acertar os acentos
\usepackage[brazilian]{babel} 
%\usepackage[portuguese]{babel} 
% \usepackage[english]{babel}
% \usepackage[T1]{fontenc}
% \usepackage[latin1]{inputenc}
\usepackage[utf8]{inputenc}
\usepackage{indentfirst}
\usepackage{fullpage}
% \usepackage{graphicx} %See PDF section
\usepackage{multicol}
\setlength{\columnseprule}{0.5pt}
\setlength{\columnsep}{20pt}
%%%%%%%%%%%%%%%%%%%%%%%%
%%%%%%%%%%%%%%%%%%%%%%%%
%% PDF support

\usepackage[pdftex]{color,graphicx}
% %% Hyper-refs
\usepackage[pdftex]{hyperref} % for printing
% \usepackage[pdftex,bookmarks,colorlinks]{hyperref} % for screen

%% \newif\ifPDF
%% \ifx\pdfoutput\undefined\PDFfalse
%% \else\ifnum\pdfoutput > 0\PDFtrue
%%      \else\PDFfalse
%%      \fi
%% \fi

%% \ifPDF
%%   \usepackage[T1]{fontenc}
%%   \usepackage{aeguill}
%%   \usepackage[pdftex]{graphicx,color}
%%   \usepackage[pdftex]{hyperref}
%% \else
%%   \usepackage[T1]{fontenc}
%%   \usepackage[dvips]{graphicx}
%%   \usepackage[dvips]{hyperref}
%% \fi

%%%%%%%%%%%%%%%%%%%%%%%%


%%%%%%%%%%%%%%%%%%%%%%%%
%% Math
\usepackage{amsmath,amsfonts,amssymb}
% para usar R de Real do jeito que o povo gosta
\usepackage{amsfonts} % \mathbb
% para usar as letras frescas como L de Espaco das Transf Lineares
% \usepackage{mathrsfs} % \mathscr

% Oferecer seno e tangente em pt, com os comandos usuais.
\providecommand{\sin}{} \renewcommand{\sin}{\hspace{2pt}\mathrm{sen}}
\providecommand{\tan}{} \renewcommand{\tan}{\hspace{2pt}\mathrm{tg}}

% dt of integrals = \ud t
\newcommand{\ud}{\mathrm{\ d}}
%%%%%%%%%%%%%%%%%%%%%%%%



\begin{document}

%%%%%%%%%%%%%%%%%%%%%%%%
%% Título e cabeçalho
%\noindent\parbox[c]{.15\textwidth}{\includegraphics[width=.15\textwidth]{logo}}\hfill
\parbox[c]{.825\textwidth}{\raggedright%
  \sffamily {\LARGE

Equações Diferenciais Ordinárias: Lista de Modelagem com Equações de
Primeira Ordem

\par\bigskip}
% {Centro Universitário Anhanguera de Niterói -- UNIAN\par} 
% {Curso: Engenharia\par}
{Prof: Felipe Figueiredo\par}
{\url{http://sites.google.com/site/proffelipefigueiredo}\par}
}

Versão: \verb|20141217|

\vspace{0.5cm}
%%%%%%%%%%%%%%%%%%%%%%%%


Observação: Você precisará usar uma calculadora na {\bf última} etapa
dos exercícios, para efetuar o cálculo da exponencial e encontrar a
resposta final. Por exemplo, se você encontrar a resposta
``$x=10e^{-1}$'', sua calculadora lhe dará a resposta final ``$3.7$''
(considerando precisão de uma casa decimal). Você deve usar as
propriedades necessárias das exponenciais e logaritmos para encontrar
a resposta antes de usar a calculadora.

Na prova, você não precisará fazer este tipo de cálculo de precisão,
portanto não será necessário (nem permitido) o uso de calculadora.

\begin{enumerate}

\section{Velocidade final}

\item Kenny\circledR\ está dirigindo de madrugada em um bairro
  perigoso que não conhece direito. Ao ver o sinal se abrir Kenny
  começa a acelerar sua {\it Ferrari}\circledR\ para percorrer a maior
  distância possível no menor tempo. O carro derrapa um pouco na
  arrancada (acordando assim alguns moradores da rua), mas após
  atingir a velocidade $v_0 = 1\frac{m}{s}$ este passa a se movimentar
  para a frente. De olho no velocímetro, ele aumenta a força no
  acelerador de modo que a aceleração aumente proporcionalmente à
  velocidade em cada instante a uma taxa de $\lambda=0,3$. Após
  acelerar dessa forma durante $t=15s$ pela rua (que para seu azar é
  mal sinalizada e mal iluminada), esta faz uma curva brusca, e ele
  atinge o muro de uma casa (acordando assim todos os moradores da
  rua). Assuma, por simplicidade, que a massa do carro é desprezível,
  e que a velocidade do mesmo aumenta apenas em função da aceleração.

  \begin{enumerate}
  \item Qual é a equação que descreve a variação da velocidade $v$ da
    {\it Ferrari} de Kenny em função do tempo $t$?
  \item Qual é a velocidade terminal (em $\frac{m}{s}$, com precisão
    de uma casa decimal) de Kenny ao encontrar sua morte?
  \item (perspectiva) Converta a velocidade acima para $\frac{km}{h}$
    e entenda porque Kenny terá um enterro com caixão fechado.
  \end{enumerate}

\item Em suas férias de 2013 Kenny\circledR\ fez um salto de
  pára-quedas a uma baixa altitude. No exato instante em que ele salta
  do avião, ele abre seu pára-quedas e um abutre passa e fura o
  equipamento. Com o pára-quedas danificado, a constante da
  resistência do ar é de apenas $\lambda=10$, e a gravidade
  $g=10\frac{m}{s^2}$ acelera seu corpo com massa $m=50kg$ até que ele
  atinja o chão. Considerando a resistência do ar, e que ele levou
  $t=10s$ tensos até encontrar sua morte, determine a velocidade com
  que Kenny se estatelou no chão.

  \begin{enumerate}
  \item Qual é a equação que descreve a variação da velocidade de
    queda $v$ de Kenny, considerando a resistência do ar $\lambda$, a
    gravidade $g$ e a massa $m$ de Kenny \cite{PLT}.
  \item Qual é a velocidade terminal de Kenny (em $\frac{m}{s}$, com
    precisão de uma casa decimal) ao se estatelar no chão?
  \item (perspectiva) Converta a velocidade acima para $\frac{km}{h}$
    e entenda porque Kenny terá um enterro com caixão fechado.
  \end{enumerate}

\section{Juros compostos}

\item Você fez um investimento $I$ em uma poupança que rende 4\% ao
  ano em juros compostos. Seu depósito inicial foi de R\$10.000.

  \begin{enumerate}
  \item Qual é a equação que descreve o aumento do seu investimento
    $I$ (em R\$) no tempo ($t$ em anos)?
  \item Qual é o saldo de sua conta após 5 anos?
  \end{enumerate}

\section{Pressão atmosférica}

\item Ao atingir a altitude de $h=10km$, o cosmonauta russo Iuri
  Gagarin em 1961 percebeu que sentia uma grande diferença na pressão
  atmosférica a bordo da nave (fnv: fato não-verídico). Ele não sabia
  qual era a densidade do ar naquela altitude, mas era muito bom em
  resolver equações diferenciais (fnv). Naquele instante ele se
  lembrou que (a) a pressão atmosférica $P$ na superfície era de $1$
  atmosfera ou, aproximadamente $100kPa$ (quilo pascals), e (b) a
  altitudes de até $11km$ a pressão atmosférica cai proporcionalmente
  com a altitude a uma taxa aproximada de $1,328kPa$ a cada
  $100m$. % supôs que a fez uma
  % estimativa simplificada %\cite{PressaoAtmosferica}

  \begin{enumerate}
  \item Qual é a constante $\lambda$ de proporcionalidade do
    decaimento da pressão atmosférica $P$ com as unidades ajustadas
    (altitude em $km$ e pressão em $kPa$)?
  \item Qual foi a equação que Gagarin utilizou para estimar a pressão
    atmosférica $P$ (em $kPa$) em função da altitude $h$ (em $km$).
  \item Qual foi a estimativa da pressão atmosférica $P$ obtida por
    Iuri à altitude $h=10km$ antes de olhar pela janela e descobrir
    que ``a Terra é azul''? Considere precisão de uma casa decimal.
  \end{enumerate}

\item (pressão de uma coluna em líquido) Em breve \ldots

 % Jacques-Cousteau certa vez fez um
 %  mergulho de x metros de profundidade (fnv). Qual era a pressão que
 %  ele sentia a essa profundidade? FIXME



\section{Diluição de uma concentração}

\item Em breve \ldots
  % Um tanque com um misturador contendo x L de solução água e sal
  % recebe água pura de uma fonte a uma taxa de $10L/s$, e perde
  % solução por uma torneira a uma taxa de $C=5L/s$. Assuma que o
  % misturador é ideal e mantém a solução homogênea durante todo o
  % processo. FIXME

%   \begin{enumerate}
%   \item Qual é a equação que determina a variação da concentração $C$
%     em função do tempo (em $s$)?
%   \item Após $60s$, qual é a concentração $C$ de sal no tanque.
%   \end{enumerate}


\section{Circuitos elétricos}

\item (Circuito RL) Uma bateria de 12v é conectada a um circuito em
  série RL com uma indutância de $L=1$ e uma resistência de
  $R=2$. Qual é a equação que descreve a variação da corrente $I$ no
  tempo, considerando que a corrente inicial era $I_0=0$?


\item (Circuito RC) Um circuito em série RC é conectado a uma fonte de
  100v com um capacitor de $C=10^{-2}$ e uma resistência de
  $R=10$. Qual é a equação que descreve a variação da quantidade de
  carga $Q$ no capacitor, considerando que a quantidade inicial era
  $Q_0=0$?



\section{Perda de temperatura para o meio}

\item (Café frio) Você prepara seu café, que ao ficar pronto está com
  temperatura $T=95^\text{o}$C. Como você está concentrado nos seus
  exercícios de EDO, o café fica esquecido por $t=15$ minutos sobre a
  mesa, exposto à temperatura ambiente. A Lei de Newton do
  Resfriamento \cite{NewtonResfriamento} diz que o café perderá calor
  para o meio a uma taxa proporcional à diferença entre sua própria
  temperatura e a temperatura do meio, tendendo a se igualar à
  temperatura do ambiente. Nesta sala, a constante de
  proporcionalidade de perda de temperatura é $\lambda = 0,1343$ e a
  temperatura ambiente é $T_a=20^\text{o}$C.

  \begin{enumerate}
  \item Qual é a equação que relaciona a temperatura do café (em
    $^\text{o}$C) em função do tempo $t$ (em minutos), considerando a
    temperatura do ambiente $T_a$ e a constante de proporcionalidade
    $\lambda$?
  \item Quando você percebe que esqueceu de seu café (depois de
    $t=15min$), qual é a temperatura final dele (em $^\text{o}$C) com
    precisão de uma casa decimal?
  \end{enumerate}

\item (CSI) Kenny\circledR foi assassinado em um escritório, onde um
  condicionador de ar mantém a temperatura constante. Você é um perito
  criminal e chega na cena de um assasinato para estimar a hora da
  morte. Você chega na cena do crime à meia noite e introduz um
  termômetro no fígado do corpo, observando que sua temperatura
  naquele momento é $T=27^o$C. Duas horas depois, você mede a
  temperatura novamente obtendo o valor $T=25^\text{o}$C. A
  temperatura da sala é regulada pelo ar condicionado e mantida em
  $T_a=19^\text{o}$C. \cite{CSI}

  \begin{enumerate}
  \item Considerando a Lei de Newton do Resfriamento, qual é a equação
    diferencial que relaciona a temperatura $T$ do corpo em função do
    tempo $t$, considerando a temperatura do meio $T_a$ constante
    entre o óbito e a constante de proporcionalidade $\lambda$?
  \item Qual é a constante de proporcionalidade $\lambda$ de troca de
    temperatura neste ambiente?
  \item Qual é o tempo aproximado desde o óbito até a primeira medição
    de temperatura? Assuma tempo $t$ em horas, temperatura $T$ em
    $^\text{o}$C, e use precisão de uma casa decimal. Assuma que a
    temperatura do corpo (vivo!) é $T=37^\text{o}$C.
  \item (perspectiva) Qual é a hora aproximada da morte?
  \end{enumerate}

\section{Decaimento radioativo}

\item (Meia vida) A meia vida $t_m$ de um isótopo radioativo é o tempo
  necessário para que sua massa original seja reduzida à metade,
  assumindo decaimento proporcional à quantidade $Q$ existente em cada
  instante $t$. A meia vida do C$^{14}$ (Carbono--14) é
  aproximadamente 5730 anos. Qual é a constante de proporcionalidade
  do decaimento $\lambda$ do C$^{14}$?


\item A usina nuclear de Chernobyl sofreu em 1986 um grave acidente,
  culminando na explosão do reator 4 (\cite{Chernobyl1,
    Chernobyl2}). Após a explosão do reator o Cs$^{137}$ (Césio--137)
  lá utilizado foi liberado e transportado pelo vento, contaminando a
  região do entorno da usina. Assuma que uma quantidade $Q=27kg$ de
  Cs$^{137}$ foi distribuída de forma homogênea na região da usina, e
  sabendo que a meia vida do Cs$^{137}$ é de $t_m=30$ anos determine:

  \begin{enumerate}
  \item Qual é a equação que descreve o decaimento da quantidade $Q$
    do material radioativo?
  \item De acordo com o modelo acima, qual é a quantidade esperada de
    Cs$^{137}$ em 2016 (em $kg$)?
  \item Kenny\circledR\ foi passear em Chernobyl em suas férias de
    2016, e morreu lenta e dolorosamente de câncer de tireóide por
    conta da radiação liberada pela quantidade de Cs$^{137}$ estimada
    acima. Em que ano ele deveria ter feito esse passeio para ser
    exposto a uma quantidade de (apenas) $1kg$?
  \end{enumerate}

\item Um colecionador milionário está interessado em comprar uma
  pintura do Rembrandt (1606--1669), e contratou você como
  consultor(a) para averiguar a autenticidade da obra. Pela análise de
  datação de carbono (feita em 2014), você aferiu que a pintura em
  questão contém 99,879\% do C$^{14}$ original.

  \begin{enumerate}
  \item Determine a idade (em anos, com precisão de duas casas
    decimais) da pintura em questão, e conclua se é ou não uma
    falsificação.
  \item Determine aproximadamente (com uma casa decimal de precisão)
    qual o percentual máximo do C$^{14}$ original que uma obra
    autêntica de Rembrandt poderia ter (sugestão: por simplificação,
    considere como cota superior o ano de morte de Rembrandt).
  \end{enumerate}
\end{enumerate}


%%%%%%%%%%%%%%%%%%%%%%%%


\begin{thebibliography}{9}
% \bibitem{PressaoAtmosferica}
%   \url{http://psas.pdx.edu/RocketScience/PressureAltitude_Derived.pdf}
%   (Acessado em 30 de Agosto de 2014)

\bibitem{PLT} Esta modelagem pode ser encontrada na explicação do
  livro texto (PLT) da disciplina.

\bibitem{NewtonResfriamento}
  \url{http://www2.pelotas.ifsul.edu.br/denise/caloretemperatura/resfriamento.pdf}
  (Acessado em 30 de Agosto de 2014)

\bibitem{CSI}
  \url{https://www.academia.edu/3467740/Forensic_Estimation_of_Time_of_Death_A_Mathematical_Model}
(Acessado em 1 de Setembro de 2014)

\bibitem{Chernobyl1} \url{http://chernobyl.undp.org/english/}
  (Acessado em 31 de Agosto de 2014)

\bibitem{Chernobyl2}
  \url{http://www.world-nuclear.org/info/safety-and-security/safety-of-plants/chernobyl-accident/}
  (Acessado em 31 de Agosto de 2014)

\end{thebibliography}

\end{document}
