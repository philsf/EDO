\everymath{\displaystyle}
%\documentclass[pdftex,a4paper]{article}
\documentclass[a4paper]{article}
%%classes: article, report, book, proc, amsproc

%%%%%%%%%%%%%%%%%%%%%%%%
%% Misc
% para acertar os acentos
\usepackage[brazilian]{babel} 
%\usepackage[portuguese]{babel} 
% \usepackage[english]{babel}
% \usepackage[T1]{fontenc}
% \usepackage[latin1]{inputenc}
\usepackage[utf8]{inputenc}
\usepackage{indentfirst}
\usepackage{fullpage}
% \usepackage{graphicx} %See PDF section
\usepackage{multicol}
\setlength{\columnseprule}{0.5pt}
\setlength{\columnsep}{20pt}
%%%%%%%%%%%%%%%%%%%%%%%%
%%%%%%%%%%%%%%%%%%%%%%%%
%% PDF support

\usepackage[pdftex]{color,graphicx}
% %% Hyper-refs
\usepackage[pdftex]{hyperref} % for printing
% \usepackage[pdftex,bookmarks,colorlinks]{hyperref} % for screen

%% \newif\ifPDF
%% \ifx\pdfoutput\undefined\PDFfalse
%% \else\ifnum\pdfoutput > 0\PDFtrue
%%      \else\PDFfalse
%%      \fi
%% \fi

%% \ifPDF
%%   \usepackage[T1]{fontenc}
%%   \usepackage{aeguill}
%%   \usepackage[pdftex]{graphicx,color}
%%   \usepackage[pdftex]{hyperref}
%% \else
%%   \usepackage[T1]{fontenc}
%%   \usepackage[dvips]{graphicx}
%%   \usepackage[dvips]{hyperref}
%% \fi

%%%%%%%%%%%%%%%%%%%%%%%%


%%%%%%%%%%%%%%%%%%%%%%%%
%% Math
\usepackage{amsmath,amsfonts,amssymb}
% para usar R de Real do jeito que o povo gosta
\usepackage{amsfonts} % \mathbb
% para usar as letras frescas como L de Espaco das Transf Lineares
% \usepackage{mathrsfs} % \mathscr

% Oferecer seno e tangente em pt, com os comandos usuais.
\providecommand{\sin}{} \renewcommand{\sin}{\hspace{2pt}\mathrm{sen}}
\providecommand{\tan}{} \renewcommand{\tan}{\hspace{2pt}\mathrm{tg}}

% dt of integrals = \ud t
\newcommand{\ud}{\mathrm{\ d}}
%%%%%%%%%%%%%%%%%%%%%%%%



\begin{document}

%%%%%%%%%%%%%%%%%%%%%%%%
%% Título e cabeçalho
%\noindent\parbox[c]{.15\textwidth}{\includegraphics[width=.15\textwidth]{logo}}\hfill
\parbox[c]{.825\textwidth}{\raggedright%
  \sffamily {\LARGE

Equações Diferenciais Ordinárias: Gabarito de Equações Lineares com
coeficientes constantes


\par\bigskip}
% {Centro Universitário Anhanguera de Niterói -- UNIAN\par} 
% {Curso: Engenharia\par}
{Prof: Felipe Figueiredo\par}
{\url{http://sites.google.com/site/proffelipefigueiredo}\par}
}

Versão: \verb|20141124|

%%%%%%%%%%%%%%%%%%%%%%%%


%%%%%%%%%%%%%%%%%%%%%%%%


\begin{enumerate}
\item 

  \begin{enumerate}
  \item $y = Ke^{2x}$
  \item $y = Ke^{15x}$
  \item $y = Ke^{-3x}$
  \item $y = Ke^{9x}$
  \item $y = Ke^{\frac{1}{2}x}$
  \item $y = Ke^{-\frac{1}{3}x}$
  \item $y = Ke^{x}$
  \item $y = Ke^{-x}$
  \item $y = Ke^{2x}$
  \item $y = Ke^{\frac{2}{3}x}$
  \item $y = Ke^{\sqrt{3}x}$
  \item $y = Ke^{2\pi x}$
  \item $y = Ke^{rx}$
  \end{enumerate}


\item 

  \begin{enumerate}
  \item $y = Ke^{2x} - \frac{1}{2}$
  \item $y = Ke^{10x} - \frac{1}{5}$
  \item $y = Ke^{-x} - 1$
  \item $y = Ke^{x} + 1$
  \item $y = Ke^{-x} + \frac{1}{2}$
  \item $y = Ke^{\frac{4}{3}x} + \frac{5}{4}$
  \item $y = Ke^{-\frac{2}{3}x} - \frac{3}{4}$
  \item $y = Ke^{rx} - \frac{a}{r}$
  \end{enumerate}

\item 

  \begin{enumerate}
  \item $y = \pm \sqrt{x + c}$
  \item $y = \pm \sqrt{x^2 + c}$
  \item $y = \frac{1}{x+c}$
  \end{enumerate}
\end{enumerate}


\end{document}
