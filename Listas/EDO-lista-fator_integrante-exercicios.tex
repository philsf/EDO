\everymath{\displaystyle}
%\documentclass[pdftex,a4paper]{article}
\documentclass[a4paper]{article}
%%classes: article, report, book, proc, amsproc

%%%%%%%%%%%%%%%%%%%%%%%%
%% Misc
% para acertar os acentos
\usepackage[brazilian]{babel} 
%\usepackage[portuguese]{babel} 
% \usepackage[english]{babel}
% \usepackage[T1]{fontenc}
% \usepackage[latin1]{inputenc}
\usepackage[utf8]{inputenc}
\usepackage{indentfirst}
\usepackage{fullpage}
% \usepackage{graphicx} %See PDF section
\usepackage{multicol}
\setlength{\columnseprule}{0.5pt}
\setlength{\columnsep}{20pt}
%%%%%%%%%%%%%%%%%%%%%%%%
%%%%%%%%%%%%%%%%%%%%%%%%
%% PDF support

\usepackage[pdftex]{color,graphicx}
% %% Hyper-refs
\usepackage[pdftex]{hyperref} % for printing
% \usepackage[pdftex,bookmarks,colorlinks]{hyperref} % for screen

%% \newif\ifPDF
%% \ifx\pdfoutput\undefined\PDFfalse
%% \else\ifnum\pdfoutput > 0\PDFtrue
%%      \else\PDFfalse
%%      \fi
%% \fi

%% \ifPDF
%%   \usepackage[T1]{fontenc}
%%   \usepackage{aeguill}
%%   \usepackage[pdftex]{graphicx,color}
%%   \usepackage[pdftex]{hyperref}
%% \else
%%   \usepackage[T1]{fontenc}
%%   \usepackage[dvips]{graphicx}
%%   \usepackage[dvips]{hyperref}
%% \fi

%%%%%%%%%%%%%%%%%%%%%%%%


%%%%%%%%%%%%%%%%%%%%%%%%
%% Math
\usepackage{amsmath,amsfonts,amssymb}
% para usar R de Real do jeito que o povo gosta
\usepackage{amsfonts} % \mathbb
% para usar as letras frescas como L de Espaco das Transf Lineares
% \usepackage{mathrsfs} % \mathscr

% Oferecer seno e tangente em pt, com os comandos usuais.
\providecommand{\sin}{} \renewcommand{\sin}{\hspace{2pt}\mathrm{sen}}
\providecommand{\tan}{} \renewcommand{\tan}{\hspace{2pt}\mathrm{tg}}

% dt of integrals = \ud t
\newcommand{\ud}{\mathrm{\ d}}
%%%%%%%%%%%%%%%%%%%%%%%%



\begin{document}

%%%%%%%%%%%%%%%%%%%%%%%%
%% Título e cabeçalho
%\noindent\parbox[c]{.15\textwidth}{\includegraphics[width=.15\textwidth]{logo}}\hfill
\parbox[c]{.825\textwidth}{\raggedright%
  \sffamily {\LARGE

Equações Diferenciais Ordinárias: Lista de Fatores Integrantes

\par\bigskip}
% {Centro Universitário Anhanguera de Niterói -- UNIAN\par} 
% {Curso: Engenharia\par}
{Prof: Felipe Figueiredo\par}
{\url{http://sites.google.com/site/proffelipefigueiredo}\par}
}

Versão: \verb|20141124|

%%%%%%%%%%%%%%%%%%%%%%%%


%%%%%%%%%%%%%%%%%%%%%%%%
\section{Formulário}

Equação Diferencial Ordinária Linear com coeficientes variáveis:

\begin{displaymath}
  y' + p(x)y = q(x)
\end{displaymath}

Fator integrante:
\begin{displaymath}
  \mu (x) = e^{\int p(x)\ud x}
\end{displaymath}

Família de soluções:
\begin{displaymath}
  y(x) = \frac{\int (\mu q) \ud x}{\mu}
\end{displaymath}
\section{Exercícios}
\begin{enumerate}
\item Encontre a família de soluções de cada uma das seguintes
  equações diferenciais:

  \begin{enumerate}
  \item $y' +y = e^x$
  \item $y' -y = e^x$
  \item $y' -y = e^{2x}$
  \item $y' -2y = e^x$
  \item $y' = 5x^4$
  \item $2y' = 3x$
  \item $y' -y = e^x + 1$
  \item $2y' +y =e^{2x} - 2$
  \item $-3y' + 6y -12 = e^{-x} + e^x$
  \item $y' + \frac{2}{5}y = 3e^{2x} + 2e^{3x}$
  \item $y' + y = x$
  \item $y' + 2xy = x$
  \item $\frac{1}{x}y' + 2y = 3$
  \item $y' +3x^2y = x^2$
  \item $y' + \cos (x) y = \cos x$
  \end{enumerate}

\newpage
\item Encontre a solução de cada PVI abaixo:

  \begin{enumerate}
  \item $y' +y = e^x, y(0)=1$
  \item $y' -y = e^x,y(0)=0$
  \item $y' -y = e^{2x}, y(0)=1$
  \item $y' -2y = e^x, y(0)=2$
  \item $y' = 5x^4, y(1)=2$
  \item $2y' = 3x, y(1)=-1$
  \item $y' -y = e^x + 1, y(0)=-1$
  \item $2y' +y =e^{2x} - 2, y(0)=1$
  \item $-3y' + 6y -12 = e^{-x} + e^x, y(0)=0$
  \item $y' + \frac{2}{5}y = 3e^{2x} + 2e^{3x}, y(0)=1$
  \item $y' + y = x, y(0)=1$
  \item $y' + 2xy = x, y(0)=1$
  \item $\frac{1}{x}y' + 2y = 3, y(0)=1$
  \item $y' +3x^2y = x^2, y(0)=1$
  \item $y' + \cos (x) y = \cos x, y(0)=0$
  \end{enumerate}

\end{enumerate}


\end{document}
