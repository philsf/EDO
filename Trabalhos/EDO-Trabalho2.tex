\everymath{\displaystyle}
%\documentclass[pdftex,a4paper]{article}
\documentclass[a4paper]{article}
%%classes: article, report, book, proc, amsproc

%%%%%%%%%%%%%%%%%%%%%%%%
%% Misc
% para acertar os acentos
\usepackage[brazilian]{babel}
%\usepackage[portuguese]{babel}
% \usepackage[english]{babel}
% \usepackage[T1]{fontenc}
% \usepackage[latin1]{inputenc}
\usepackage[utf8]{inputenc}
\usepackage{indentfirst}
\usepackage{fullpage}
% \usepackage{graphicx} %See PDF section
\usepackage{multicol}
\setlength{\columnseprule}{0.5pt}
\setlength{\columnsep}{20pt}
%%%%%%%%%%%%%%%%%%%%%%%%
%%%%%%%%%%%%%%%%%%%%%%%%
%% PDF support

\usepackage[pdftex]{color,graphicx}
% %% Hyper-refs
\usepackage[pdftex]{hyperref} % for printing
% \usepackage[pdftex,bookmarks,colorlinks]{hyperref} % for screen

%% \newif\ifPDF
%% \ifx\pdfoutput\undefined\PDFfalse
%% \else\ifnum\pdfoutput > 0\PDFtrue
%%      \else\PDFfalse
%%      \fi
%% \fi

%% \ifPDF
%%   \usepackage[T1]{fontenc}
%%   \usepackage{aeguill}
%%   \usepackage[pdftex]{graphicx,color}
%%   \usepackage[pdftex]{hyperref}
%% \else
%%   \usepackage[T1]{fontenc}
%%   \usepackage[dvips]{graphicx}
%%   \usepackage[dvips]{hyperref}
%% \fi

%%%%%%%%%%%%%%%%%%%%%%%%


%%%%%%%%%%%%%%%%%%%%%%%%
%% Math
\usepackage{amsmath,amsfonts,amssymb}
% para usar R de Real do jeito que o povo gosta
\usepackage{amsfonts} % \mathbb
% para usar as letras frescas como L de Espaco das Transf Lineares
% \usepackage{mathrsfs} % \mathscr

% Oferecer seno e tangente em pt, com os comandos usuais.
\providecommand{\sin}{} \renewcommand{\sin}{\hspace{2pt}\mathrm{sen}}
\providecommand{\tan}{} \renewcommand{\tan}{\hspace{2pt}\mathrm{tg}}

% dt of integrals = \ud t
\newcommand{\ud}{\mathrm{\ d}}
%%%%%%%%%%%%%%%%%%%%%%%%

\date{
\bigskip
Curso: \underline{\hspace{8cm}}\\
\ \\
Turma: \underline{\hspace{1cm}} Série: \underline{\hspace{1cm}} Turno:
\underline{\hspace{1cm}}\\
\ \\
Prof: \underline{\hspace{8cm}}\\
}

\title{Equações Diferenciais - Trabalho 2}

\author{
{\bf Grupo}\\
\ \\
Nome: \underline{\hspace{6cm}} RA: \underline{\hspace{2cm}} Assinatura: \underline{\hspace{4cm}}\\
Nome: \underline{\hspace{6cm}} RA: \underline{\hspace{2cm}} Assinatura: \underline{\hspace{4cm}}\\
Nome: \underline{\hspace{6cm}} RA: \underline{\hspace{2cm}} Assinatura: \underline{\hspace{4cm}}\\
Nome: \underline{\hspace{6cm}} RA: \underline{\hspace{2cm}} Assinatura: \underline{\hspace{4cm}}\\
Nome: \underline{\hspace{6cm}} RA: \underline{\hspace{2cm}} Assinatura: \underline{\hspace{4cm}}\\
Nome: \underline{\hspace{6cm}} RA: \underline{\hspace{2cm}} Assinatura: \underline{\hspace{4cm}}\\
Nome: \underline{\hspace{6cm}} RA: \underline{\hspace{2cm}} Assinatura: \underline{\hspace{4cm}}\\
Nome: \underline{\hspace{6cm}} RA: \underline{\hspace{2cm}} Assinatura: \underline{\hspace{4cm}}\\
Nome: \underline{\hspace{6cm}} RA: \underline{\hspace{2cm}} Assinatura: \underline{\hspace{4cm}}\\
Nome: \underline{\hspace{6cm}} RA: \underline{\hspace{2cm}} Assinatura: \underline{\hspace{4cm}}\\
}

% {Aluno(a): \underline{\hspace{8.5cm}} RA: \underline{\hspace{2.3cm}}\par}
% {Curso: \underline{\hspace{8.95cm}} Data: \underline{\hspace{2cm}}\par}
% {Turma: \underline{\hspace{2cm}} Série: \underline{\hspace{2cm}}
%   Turno: \underline{\hspace{2.5cm}} Prova A\par}
% {Prof: \underline{\hspace{9.15cm}} Nota: \underline{\hspace{2cm}}\par}

\begin{document}
\maketitle

\newpage
%%%%%%%%%%%%%%%%%%%%%%%%
%% Título e cabeçalho
%\noindent\parbox[c]{.15\textwidth}{\includegraphics[width=.15\textwidth]{logo}}\hfill
\parbox[c]{.825\textwidth}{\raggedright%
  \sffamily {\LARGE

Equações Diferenciais: Trabalho 2

\par\bigskip}
{Prof: Felipe Figueiredo\par}
{\url{http://sites.google.com/site/proffelipefigueiredo}\par}
}

Versão: \verb|20150929|

%%%%%%%%%%%%%%%%%%%%%%%%


%%%%%%%%%%%%%%%%%%%%%%%%

\section{Valor}
Este trabalho valerá $2.0$pts na composição da nota da P2. Cada 4
questões resolvidas corretamente e com resolução completa pontuam em
$0.5$pt.

\section{Entrega}

Todas resoluções devem ser feitas a lápis, e as respostas a caneta
(azul ou preta). Todas as pessoas do grupo devem assinar a capa do
trabalho.

O(a) representante de turma, ou alguém apontado por ele(a), deverá
juntar todos os trabalhos da turma. O trabalho 2 será recolhida na
aula da semana determinada no endereço:

\url{https://sites.google.com/site/proffelipefigueiredo/anhanguera/2015-2}.

\section{Conteúdo}

O grupo deve consultar no PLT da disciplina os enunciados dos
seguintes exercícios e problemas abaixo:

\begin{enumerate}
\item Seção 11.10 (pg 441) questões 4 e 18.
\item Seção 11.11 (pg 447) questões 30, 32 e 34.
\item Seção 9.1   (pg 337) questões 20, 25, 26 e 27, (pg 352) questão 18.
\item Seção 10.1  (pg 360) questões 2, 22 e 31.
\item Seção 10.5  (pg 383) questões 6, 12 e 14.
\end{enumerate}

\section{Observações}

\begin{itemize}
\item A assinatura de cada aluno na capa do trabalho é
  obrigatória. Alunos que não assinarem o trabalho, não receberão os
  pontos dessa atividade.
\item Cada questão deve ter resolução completa, ou não será
  considerada.
\item As folhas com as questões resolvidas devem ser grampeadas junto
  com a capa do trabalho.
\end{itemize}

\end{document}
