\everymath{\displaystyle}
%\documentclass[pdftex,a4paper]{article}
\documentclass[a4paper]{article}
%%classes: article, report, book, proc, amsproc

%%%%%%%%%%%%%%%%%%%%%%%%
%% Misc
% para acertar os acentos
\usepackage[brazilian]{babel} 
%\usepackage[portuguese]{babel} 
% \usepackage[english]{babel}
% \usepackage[T1]{fontenc}
% \usepackage[latin1]{inputenc}
\usepackage[utf8]{inputenc}
\usepackage{indentfirst}
\usepackage{fullpage}
% \usepackage{graphicx} %See PDF section
\usepackage{multicol}
\setlength{\columnseprule}{0.5pt}
\setlength{\columnsep}{20pt}
%%%%%%%%%%%%%%%%%%%%%%%%
%%%%%%%%%%%%%%%%%%%%%%%%
%% PDF support

\usepackage[pdftex]{color,graphicx}
% %% Hyper-refs
\usepackage[pdftex]{hyperref} % for printing
% \usepackage[pdftex,bookmarks,colorlinks]{hyperref} % for screen

%% \newif\ifPDF
%% \ifx\pdfoutput\undefined\PDFfalse
%% \else\ifnum\pdfoutput > 0\PDFtrue
%%      \else\PDFfalse
%%      \fi
%% \fi

%% \ifPDF
%%   \usepackage[T1]{fontenc}
%%   \usepackage{aeguill}
%%   \usepackage[pdftex]{graphicx,color}
%%   \usepackage[pdftex]{hyperref}
%% \else
%%   \usepackage[T1]{fontenc}
%%   \usepackage[dvips]{graphicx}
%%   \usepackage[dvips]{hyperref}
%% \fi

%%%%%%%%%%%%%%%%%%%%%%%%


%%%%%%%%%%%%%%%%%%%%%%%%
%% Math
\usepackage{amsmath,amsfonts,amssymb}
% para usar R de Real do jeito que o povo gosta
\usepackage{amsfonts} % \mathbb
% para usar as letras frescas como L de Espaco das Transf Lineares
% \usepackage{mathrsfs} % \mathscr

% Oferecer seno e tangente em pt, com os comandos usuais.
\providecommand{\sin}{} \renewcommand{\sin}{\hspace{2pt}\mathrm{sen}}
\providecommand{\tan}{} \renewcommand{\tan}{\hspace{2pt}\mathrm{tg}}

% dt of integrals = \ud t
\newcommand{\ud}{\mathrm{\ d}}
%%%%%%%%%%%%%%%%%%%%%%%%



\begin{document}

%%%%%%%%%%%%%%%%%%%%%%%%
%% Título e cabeçalho
%\noindent\parbox[c]{.15\textwidth}{\includegraphics[width=.15\textwidth]{logo}}\hfill
\parbox[c]{.825\textwidth}{\raggedright%
  \sffamily {\LARGE

Equações Diferenciais: Notas de Aula

EDOs de 2a ordem homogêneas

\par\bigskip}
{Prof: Felipe Figueiredo\par}
{\url{http://sites.google.com/site/proffelipefigueiredo}\par}
}

Versão: \verb|20151015|

%%%%%%%%%%%%%%%%%%%%%%%%


%%%%%%%%%%%%%%%%%%%%%%%%
\section{Objetivos de aprendizagem}

Ao final desta aula o aluno deve saber resolver uma EDO de segunda
ordem, com coeficientes constantes, usando o método da equação
característica, e resolver PVIs e PVCs homogêneos.


\section{Pré-requitos da aula}

\begin{itemize}
\item Raízes do polinômio do segundo grau
\item Parte real e parte imaginária de um número complexo
\end{itemize}

\section{Conteúdo}

O aluno deve consultar o livro texto na seção 11.11 para se aprofundar
no conteúdo desta aula.

% \subsection{Problema}

\subsection{O método da Equação Característica}

Considere a seguinte EDO linear de segunda ordem homogênea:

\begin{displaymath}
  ay'' + by' + cy =0
\end{displaymath}

Como ela tem os coeficientes ($a$, $b$ e $c$) constantes, podemos
resolvê-la usando sua equação característica. Para identificá-la,
basta construir um polinômio com os mesmos coeficientes que a EDO,
``substituindo'' as derivadas pelos expoentes:

\begin{displaymath}
  ar^2 + br +c=0
\end{displaymath}

Assim, a segunda derivada $y''$ está associada a $r^2$, e assim
sucessivamente. Observe que a função $y$ não tem derivadas, por isto
está associada a $r^0=1$.

Para resolver o polinômio, usamos a tradicional fórmula de Bhaskara:

\begin{displaymath}
  r=\frac{-b \pm \sqrt{\Delta}}{2a}
\end{displaymath}
onde
\begin{displaymath}
  \Delta = b^2 - 4ac
\end{displaymath}

Fazendo o estudo dos sinais de $\Delta$, podemos descobrir como são as
raízes. Se $\Delta>0$, a equação possui duas raízes reais distintas
($r_1$ e $r_2$). Se $\Delta=0$, ela possui duas raízes reais iguais
($r_1=r_2=r$). Quando $\Delta<0$, a equação característica possui duas
raízes complexas conjugadas ($\alpha \pm \beta i$).

Estas três situações possíveis nos retornam três fórmulas para a
solução geral da EDO de segunda ordem:

\begin{enumerate}
\item $\Delta>0$
  \begin{displaymath}
    y(x) = K_1e^{r_1x} + K_2e^{r_2x}
  \end{displaymath}
\item $\Delta=0$
  \begin{displaymath}
    y(x) = K_1e^{rx} + K_2xe^{rx}
  \end{displaymath}
\item $\Delta<0$
  \begin{displaymath}
    y(x) = K_1e^{\alpha x}\cos(\beta x) + K_2e^{\alpha x}\sin(\beta x)
  \end{displaymath}
\end{enumerate}

\subsection{Exemplos}

\begin{multicols}{2}
{\bf Exemplo 1}
Considere o seguinte PVI:

\begin{displaymath}
  y''-y=0, y(0)=1, y'(0)=3
\end{displaymath}

Equação característica:
\begin{displaymath}
  r^2-1=0
\end{displaymath}

Como $\Delta>0$, existem duas raízes reais distintas:

\begin{displaymath}
  r_1 = 1, r_2=-1
\end{displaymath}

Substituindo na fórmula, encontramos a solução geral da EDO:

\begin{displaymath}
  y(x) = K_1e^x+K_2e^{-x}
\end{displaymath}

Para encontrar o valor das constantes $K$, usamos as condições
iniciais ($y_0$ e $y'_0$):

\begin{displaymath}
  y(0) = K_1e^0+K_2e^0 = K_1 + K_2
\end{displaymath}
\begin{displaymath}
  1=K_1+K_2
\end{displaymath}

Para usar a condição $y'(0)$, precisamos calcular a derivada de $y$,
que é:
\begin{displaymath}
  y'(x) = K_1e^x-K_2e^{-x}
\end{displaymath}

Substituindo:
\begin{displaymath}
    y'(0) = K_1e^0-K_2e^0
\end{displaymath}
\begin{displaymath}
  3=K_1-K_2
\end{displaymath}

Temos assim, um sistema $2 \times 2$ para as constantes $K$:
\begin{displaymath}
  \left\{
    \begin{array}{l}
      K_1+K_2=1\\
      K_1-K_2=3
    \end{array}
\right.
\end{displaymath}

Resolvendo, encontramos $K_1=2$ e $K_2=-1$. Substituindo estes valores
na solução geral, encontramos a solução do PVI:

\begin{displaymath}
  y(x) =  2e^x -e^{-x}
\end{displaymath}

\hrulefill

{\bf Exemplo 2}

Considere o seguinte PVI:
\begin{displaymath}
  y''-4y' +4y=0, y(0)=3, y'(0)=10
\end{displaymath}

Equação característica

\begin{displaymath}
  r^2-4r+4=0
\end{displaymath}

Discriminante: $\Delta=0$ (uma raiz real $r$).

Raiz: $r=2$

Solução geral: $y(x) = K_1e^{2x} + K_2xe^{2x}$

Substituindo a condição inicial $y(0)$:

\begin{displaymath}
  y(0) = K_1e^0 + 0 \Rightarrow K_1=3
\end{displaymath}

Derivada: $y'(x) = 2K_1e^{2x} + K_2e^{2x} + 2K_2xe^{2x}$

Substituindo a condição inicial $y'(0)$:
\begin{displaymath}
  y'(0) = 2\times 3\times e^0 + K_2e^0 + 0 \Rightarrow 6+ K_2=10
\end{displaymath}

Portanto $K_1=3$ e $K_2=4$. A solução do PVI é portanto:

\begin{displaymath}
  y(x) = 3e^{2x} + 4xe^{2x}
\end{displaymath}


\end{multicols}
\end{document}
