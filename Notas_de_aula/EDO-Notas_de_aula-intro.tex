\everymath{\displaystyle}
%\documentclass[pdftex,a4paper]{article}
\documentclass[a4paper]{article}
%%classes: article, report, book, proc, amsproc

%%%%%%%%%%%%%%%%%%%%%%%%
%% Misc
% para acertar os acentos
\usepackage[brazilian]{babel} 
%\usepackage[portuguese]{babel} 
% \usepackage[english]{babel}
% \usepackage[T1]{fontenc}
% \usepackage[latin1]{inputenc}
\usepackage[utf8]{inputenc}
\usepackage{indentfirst}
\usepackage{fullpage}
% \usepackage{graphicx} %See PDF section
\usepackage{multicol}
\setlength{\columnseprule}{0.5pt}
\setlength{\columnsep}{20pt}
%%%%%%%%%%%%%%%%%%%%%%%%
%%%%%%%%%%%%%%%%%%%%%%%%
%% PDF support

\usepackage[pdftex]{color,graphicx}
% %% Hyper-refs
\usepackage[pdftex]{hyperref} % for printing
% \usepackage[pdftex,bookmarks,colorlinks]{hyperref} % for screen

%% \newif\ifPDF
%% \ifx\pdfoutput\undefined\PDFfalse
%% \else\ifnum\pdfoutput > 0\PDFtrue
%%      \else\PDFfalse
%%      \fi
%% \fi

%% \ifPDF
%%   \usepackage[T1]{fontenc}
%%   \usepackage{aeguill}
%%   \usepackage[pdftex]{graphicx,color}
%%   \usepackage[pdftex]{hyperref}
%% \else
%%   \usepackage[T1]{fontenc}
%%   \usepackage[dvips]{graphicx}
%%   \usepackage[dvips]{hyperref}
%% \fi

%%%%%%%%%%%%%%%%%%%%%%%%


%%%%%%%%%%%%%%%%%%%%%%%%
%% Math
\usepackage{amsmath,amsfonts,amssymb}
% para usar R de Real do jeito que o povo gosta
\usepackage{amsfonts} % \mathbb
% para usar as letras frescas como L de Espaco das Transf Lineares
% \usepackage{mathrsfs} % \mathscr

% Oferecer seno e tangente em pt, com os comandos usuais.
\providecommand{\sin}{} \renewcommand{\sin}{\hspace{2pt}\mathrm{sen}}
\providecommand{\tan}{} \renewcommand{\tan}{\hspace{2pt}\mathrm{tg}}

% dt of integrals = \ud t
\newcommand{\ud}{\mathrm{\ d}}
%%%%%%%%%%%%%%%%%%%%%%%%



\begin{document}

%%%%%%%%%%%%%%%%%%%%%%%%
%% Título e cabeçalho
%\noindent\parbox[c]{.15\textwidth}{\includegraphics[width=.15\textwidth]{logo}}\hfill
\parbox[c]{.825\textwidth}{\raggedright%
  \sffamily {\LARGE

Equações Diferenciais: Notas de Aula: Introdução às EDOs

\par\bigskip}
{Prof: Felipe Figueiredo\par}
{\url{http://sites.google.com/site/proffelipefigueiredo}\par}
}

Versão: \verb|20150723|

%%%%%%%%%%%%%%%%%%%%%%%%


%%%%%%%%%%%%%%%%%%%%%%%%
\section{Objetivos de aprendizagem}

Ao final desta aula o aluno deve saber identificar uma Equação
Diferencial, em comparação aos tipos de equações elementares já
familiares ao aluno.

\begin{itemize}
\item Breve revisão de derivadas
\item Notações e nomenclaturas de EDOs
\item Testar candidato a solução
\end{itemize}


\section{Pré-requitos da aula}

\begin{itemize}
\item Derivação das funções polinomiais e exponenciais
\end{itemize}

Exemplos de trabalho: derivar
\begin{itemize}
\item $y=x^2$
\item $y=5x^3$
\item $y=e^{2x}$
\item $y=2e^{2x}$
\end{itemize}

% \subsection{Derivadas}

% Quando existe, a derivada de uma função $f(x)$ no ponto $x_0$ é o
% valor $f'(x_0)$ que corresponde à variação instantânea de $f$ no ponto
% $x_0$. Ao variar o ponto $x_0$ para outro ponto $x_1$, obtemos o valor
% $f'(x_1)$, possivelmente diferente. Assim sucessivamente, se
% percorrermos todos os valores possíveis para

\section{Conteúdo}

\subsection{Tipos de equações já familiares}

Equações algébricas, envolvem apenas operações algébricas na
variável. São equações em que a variável representa um número.

\subsubsection{Equação de primeiro grau}

Equação: $3x -6 = 0$

Verificar candidato a solução: $x=2$

{\bf Resolução} (observar o lado esquerdo da igualdade)

$3(2) - 6 =0$

$ 6-6 =0$

$0 = 0$

\newpage
\subsubsection{Equação de segundo grau}

Equação: $x^2 - 3x + 2 = 0$

Verificar candidato a solução: $x_1 = 1, x_2 = 3$

{\bf Resolução:}
\begin{multicols}{2}
$x_1$:

$1^2 -3(1) + 2 = 0$

$1 - 3 + 2 = 0$

$-2 + 2 = 0$

$0=0$

\columnbreak

$x_2$:

$3^2 -3(3) +2 = 0$

$9-9+2 = 0 $

$0 + 2 = 0$

$2=0$

\end{multicols}

\subsubsection{Sistema de equações lineares}

Sistema: $
\left\{ \begin{array}{c}
    2x_1 + 4x_2 = 10\\
    3x_1 + 4x_2 = 12\\
  \end{array}
\right.$

Verificar candidato a solução: $\{x_1= 1, x_2=2\}$

{\bf Resolução:}

\begin{multicols}{2}
Primeira equação:

$2(1) +4(2)=10$

$2+8=10$

$10=10$

\columnbreak
Segunda equação:

$3(1)+4(2)=12$

$3+8=12$

$11=12$
\end{multicols}

\subsubsection{Equação exponencial}

Equação: $2^x = 4$

Verificar candidato a solução: $x=3$

{\bf Resolução:}

$2^2 = 4$

$4=4$

% \subsubsection{Equação trigonométrica}

% Equação: $\sin x + \cos x = -1$

% Verificar candidato a solução: $x = \frac{3\pi}{2}$

\subsection{Equações Diferenciais}

Equações em que a variável representa uma função, $y=y(x)$. Nesta
equação aparecem tanto a função $y$ como suas derivadas $y'$, $y''$,
etc.

Equação: $y' = y + 1$

Equação: $y' = y-y^2$

Como verificar candidatos a soluções nesse caso? Como sempre, basta
substituir na equação. Para isso, precisaremos derivar a função
quantas vezes for necessário.

\subsubsection{Exemplos}

Equação: $y' - 3y = 0$

Testar os seguintes candidatos de solução: $y=x^3$, $y=e^{3x}$ e $y=2e^{3x}$.

% Testar candidato de solução: $y=e^{3x}$

% Testar candidato de solução: $y=2e^{3x}$

{\bf Resolução:}

Primeiramente vamos reescrever a equação como:

$y' - 3y = 0$

\begin{multicols}{3}

$y=e^{3x}$

$y' = 3e^{3x}$

Substituindo na equação:

$3x^{3x} - 3(e^{3x}) = 0$

$3x^{3x} - 3x^{3x} =0$

$0=0$

Statisfaz para todo $x$

\columnbreak

$y=2e^{3x}$

$y' = 6e^{3x}$

Substituindo na equação:

$6e^{3x} - 2(3e^{3x}) = 0$

$6e^{3x} - 6e^{3x} =0$

$0=0$

Statisfaz para todo $x$

\columnbreak

$y=x^3$

$y' = 3x^2$

Substituindo na equação:

$3x^2 - 3(x^3) =0 $

$3x^2 - 3x^3 =0 $

Não satisfaz para todo $x$!

\end{multicols}



\subsubsection{Exercício}

Equação: $y'' + 4y = 0$

Testar: $y=e^{2x}$

\end{document}
