\everymath{\displaystyle}
%\documentclass[pdftex,a4paper]{article}
\documentclass[a4paper]{article}
%%classes: article, report, book, proc, amsproc

%%%%%%%%%%%%%%%%%%%%%%%%
%% Misc
% para acertar os acentos
\usepackage[brazilian]{babel} 
%\usepackage[portuguese]{babel} 
% \usepackage[english]{babel}
% \usepackage[T1]{fontenc}
% \usepackage[latin1]{inputenc}
\usepackage[utf8]{inputenc}
\usepackage{indentfirst}
\usepackage{fullpage}
% \usepackage{graphicx} %See PDF section
\usepackage{multicol}
\setlength{\columnseprule}{0.5pt}
\setlength{\columnsep}{20pt}
%%%%%%%%%%%%%%%%%%%%%%%%
%%%%%%%%%%%%%%%%%%%%%%%%
%% PDF support

\usepackage[pdftex]{color,graphicx}
% %% Hyper-refs
\usepackage[pdftex]{hyperref} % for printing
% \usepackage[pdftex,bookmarks,colorlinks]{hyperref} % for screen

%% \newif\ifPDF
%% \ifx\pdfoutput\undefined\PDFfalse
%% \else\ifnum\pdfoutput > 0\PDFtrue
%%      \else\PDFfalse
%%      \fi
%% \fi

%% \ifPDF
%%   \usepackage[T1]{fontenc}
%%   \usepackage{aeguill}
%%   \usepackage[pdftex]{graphicx,color}
%%   \usepackage[pdftex]{hyperref}
%% \else
%%   \usepackage[T1]{fontenc}
%%   \usepackage[dvips]{graphicx}
%%   \usepackage[dvips]{hyperref}
%% \fi

%%%%%%%%%%%%%%%%%%%%%%%%


%%%%%%%%%%%%%%%%%%%%%%%%
%% Math
\usepackage{amsmath,amsfonts,amssymb}
% para usar R de Real do jeito que o povo gosta
\usepackage{amsfonts} % \mathbb
% para usar as letras frescas como L de Espaco das Transf Lineares
% \usepackage{mathrsfs} % \mathscr

% Oferecer seno e tangente em pt, com os comandos usuais.
\providecommand{\sin}{} \renewcommand{\sin}{\hspace{2pt}\mathrm{sen}}
\providecommand{\tan}{} \renewcommand{\tan}{\hspace{2pt}\mathrm{tg}}

% dt of integrals = \ud t
\newcommand{\ud}{\mathrm{\ d}}
%%%%%%%%%%%%%%%%%%%%%%%%



\begin{document}

%%%%%%%%%%%%%%%%%%%%%%%%
%% Título e cabeçalho
%\noindent\parbox[c]{.15\textwidth}{\includegraphics[width=.15\textwidth]{logo}}\hfill
\parbox[c]{.825\textwidth}{\raggedright%
  \sffamily {\LARGE

Equações Diferenciais: Notas de Aula

EDOs de 1a ordem Separáveis

\par\bigskip}
{Prof: Felipe Figueiredo\par}
{\url{http://sites.google.com/site/proffelipefigueiredo}\par}
}

Versão: \verb|20150825|

%%%%%%%%%%%%%%%%%%%%%%%%


%%%%%%%%%%%%%%%%%%%%%%%%
\section{Objetivos de aprendizagem}

Ao final desta aula, o aluno deve saber reconhecer se uma EDO é
separável, e resolvê-la usando o Método de Separação de Variáveis.

\section{Pré-requisitos da aula}

\begin{multicols}{2}
\begin{itemize}
\item Propriedades de exponenciais:
  \begin{itemize}
  \item $2^{2+3} = 2^22^3$ e $5^{2+3} = 5^25^3$
  \item $e^{x+c} = e^xe^c = e^ce^x$
  \end{itemize}
\item Logaritmo e exponencial são funções inversas uma da outra (mesma
  base!):
  \begin{itemize}
  \item $2^{\log_2 3} = 3$
  \item $5^{\log_5 12} = 12$
  \item $e^{\ln u} = u$
  \end{itemize}
\item Primitiva de $\frac{1}{u}$
  \begin{itemize}
  \item $\int \frac{1}{u} \ud u = \int \frac{\ud u}{u} = \ln u$
  \end{itemize}
\end{itemize}
\end{multicols}
\section{Conteúdo}

O aluno deve consultar o livro texto na seção 11.4 para se aprofundar
no conteúdo desta aula.

\subsection{Problema}

Pelo que foi visto na primeira aula sabemos verificar que as soluções
da equação $\frac{\ud y}{\ud x} = -\frac{x}{y}$ são círculos da forma
$x^2 + y^2 = K$. Como encontrar essa solução?

\subsection{Equações Separáveis}

{\bf Primeiro exemplo:}
Equação: $\frac{\ud y}{\ud x} - y = 0$

\begin{multicols}{2}
{\bf Resolução:}

Arrumando a equação, temos:

\smallskip

$\frac{\ud y}{\ud x} = y$

\smallskip

Separando as variáveis, temos:

$\frac{\ud y}{y} = 1 \ud x$

\smallskip

Integrando o lado esquerdo para $y$ e o lado direito para $x$ temos:

$\int \frac{\ud y}{y} = \int \ud x \Rightarrow \ln y = x+c$

\smallskip

Aplicando a exponencial para eliminar o logaritmo:

$y = e^{x+c}$

\smallskip

Pela propriedade da exponencial:

$y = e^xe^c$

\smallskip

Como $e^c$ é uma constante, vamos renomeá-la, chamando de $K$:

$y=Ke^x$

\smallskip

Que é a {\em família de soluções} da equação. Qualquer valor de $K$ dá
uma função diferente que também é solução da equação.
\end{multicols}

% Observe que as equações $\frac{\ud y}{\ud x} -y = x$ e $\frac{\ud
%   y}{\ud x} = xy + x$ {\bf não são} separáveis.
Observe que as equações $y' -y = x$ e $y' = xy + 1$ {\bf não são}
separáveis.

\smallskip

Uma equação é separável se você puder escrevê-la como $y' = f(x)g(y)$
(produto!).

\smallskip

No primeiro exemplo, podemos chamar $f(x)=1$ (função constante, não
varia com $x$) e $g(y)=y$, portanto $y'=1y$ é separável.

\newpage
{\bf Outros exemplos:}

\begin{multicols}{2}

Equação: $\frac{\ud y}{\ud x} - 2y =0$

{\bf Resolução:}

(arrumando) $\frac{\ud y}{\ud x} = 2y$

(separando) $\frac{\ud y}{y} = 2\ud x$

(integrando) $\ln y = 2x +c$

(exponencial) $y = e^{2x+c}=e^{2x}e^c$

(constante) $y=Ke^{2x}$
\columnbreak

Equação: $\frac{\ud y}{\ud x}  +5y =0$

{\bf Resolução:}

(arrumando) $\frac{\ud y}{\ud x} = -5y$

(separando) $\frac{\ud y}{y} = -5\ud x$

(integrando) $\ln y = -5x +c$

(exponencial) $y = e^{-5x+c}=e^{-5x}e^c$

(constante) $y=Ke^{-5x}$

\end{multicols}

E quando a equação original não for igual a 0? E se for igual a outro
número qualquer?

{\bf Exemplos:}

\begin{multicols}{2}

Equação: $y' - y =2$

{\bf Resolução:}

(arrumando) $y' = y + 2$

(separando) $\frac{y'}{y+2} = 1$

\smallskip

(integrando) $\ln (y+2) = x+c$

(exponencial) $y+2 = e^{x+c}=e^{x}e^c$

(constante) $y+2=Ke^{x}$

(resposta) $y=Ke^{x}-2$

\columnbreak

Equação: $y' + y =-1$

{\bf Resolução:}

(arrumando) $y' = - y-1 = -(y+1)$

(separando) $\frac{y'}{y+1} = -1$

\smallskip

(integrando) $\ln (y+1) = -x+c$

(exponencial) $y+1 = e^{x+c}=e^{-x}e^c$

(constante) $y+1=Ke^{-x}$

(resposta) $y=Ke^{-x}-1$

\end{multicols}

Por que foi necessário arrumar a última equação daquela forma, antes
de começar a resolvê-la? Vamos ver esse truque na próxima equação, com
duas resoluções diferentes.

{\bf Equação:}
$y' - 2y = 6$

\begin{multicols}{2}
{\bf Modo 1}

\smallskip

$\frac{\ud y}{\ud x} = 2y + 6$ (arrumando)

\smallskip

$\frac{\ud y}{2y+6} = 1\ud x$ (separando)

\smallskip
Precisamos da substituição:

$u = 2y+6$

\smallskip

$\ud u=2\ud y \Rightarrow \ud y = \frac{\ud u}{2}$

\smallskip

$\int \frac{1}{u} \frac{\ud u}{2} = \int 1 \ud x \Rightarrow \frac{1}{2}\int
\frac{\ud u}{u} = \int \ud x$

\smallskip

$\int \frac{\ud u}{u} = 2\int \ud x$

\smallskip

$\ln (u) = 2x+c$ (integrando)

\smallskip

Voltando para $y$, temos:

\smallskip

$\ln(2y+6)= 2x+c$

\smallskip

$2y+6 = e^{2x+c}=e^{2x}e^c$ (exponencial)

\smallskip

$2y+6=K_1e^{2x}$ (a constante $K_1 = e^c$)

\smallskip

$2y=K_1e^{2x} -6$

\smallskip

Dividindo por 2 e chamando $K = \frac{K_1}{2}$

\smallskip

$y=\frac{K_1}{2}e^{2x} -3 \Rightarrow y=Ke^{2x}-3$

\smallskip

(Ufa!)

\columnbreak
{\bf Modo 2}

\smallskip

$\frac{\ud y}{\ud x} = 2y + 6 = 2(y+3)$ (arrumando)

\smallskip

$\frac{\ud y}{y+3} = 2\ud x$ (separando)

\smallskip

$\ln (y+3) = 2x+c$ (integrando)

\smallskip

$y+3 = e^{2x+c} = e^{2x}e^c$ (exponencial)

\smallskip

$y +3 = Ke^{2x}$ (constante)
\smallskip

$y = Ke^{2x} - 3$ (resposta)
\end{multicols}


\bigskip
Conclusão: Os dois modos são igualmente válidos. Qual você prefere?

\end{document}
