\everymath{\displaystyle}
%\documentclass[pdftex,a4paper]{article}
\documentclass[a4paper]{article}
%%classes: article, report, book, proc, amsproc

%%%%%%%%%%%%%%%%%%%%%%%%
%% Misc
% para acertar os acentos
\usepackage[brazilian]{babel} 
%\usepackage[portuguese]{babel} 
% \usepackage[english]{babel}
% \usepackage[T1]{fontenc}
% \usepackage[latin1]{inputenc}
\usepackage[utf8]{inputenc}
\usepackage{indentfirst}
\usepackage{fullpage}
% \usepackage{graphicx} %See PDF section
\usepackage{multicol}
\setlength{\columnseprule}{0.5pt}
\setlength{\columnsep}{20pt}
%%%%%%%%%%%%%%%%%%%%%%%%
%%%%%%%%%%%%%%%%%%%%%%%%
%% PDF support

\usepackage[pdftex]{color,graphicx}
% %% Hyper-refs
\usepackage[pdftex]{hyperref} % for printing
% \usepackage[pdftex,bookmarks,colorlinks]{hyperref} % for screen

%% \newif\ifPDF
%% \ifx\pdfoutput\undefined\PDFfalse
%% \else\ifnum\pdfoutput > 0\PDFtrue
%%      \else\PDFfalse
%%      \fi
%% \fi

%% \ifPDF
%%   \usepackage[T1]{fontenc}
%%   \usepackage{aeguill}
%%   \usepackage[pdftex]{graphicx,color}
%%   \usepackage[pdftex]{hyperref}
%% \else
%%   \usepackage[T1]{fontenc}
%%   \usepackage[dvips]{graphicx}
%%   \usepackage[dvips]{hyperref}
%% \fi

%%%%%%%%%%%%%%%%%%%%%%%%


%%%%%%%%%%%%%%%%%%%%%%%%
%% Math
\usepackage{amsmath,amsfonts,amssymb}
% para usar R de Real do jeito que o povo gosta
\usepackage{amsfonts} % \mathbb
% para usar as letras frescas como L de Espaco das Transf Lineares
% \usepackage{mathrsfs} % \mathscr

% Oferecer seno e tangente em pt, com os comandos usuais.
\providecommand{\sin}{} \renewcommand{\sin}{\hspace{2pt}\mathrm{sen}}
\providecommand{\tan}{} \renewcommand{\tan}{\hspace{2pt}\mathrm{tg}}

% dt of integrals = \ud t
\newcommand{\ud}{\mathrm{\ d}}
%%%%%%%%%%%%%%%%%%%%%%%%



\begin{document}

%%%%%%%%%%%%%%%%%%%%%%%%
%% Título e cabeçalho
%\noindent\parbox[c]{.15\textwidth}{\includegraphics[width=.15\textwidth]{logo}}\hfill
\parbox[c]{.825\textwidth}{\raggedright%
  \sffamily {\LARGE

Equações Diferenciais: Notas de Aula

Problemas de Valor Inicial (PVI) e de Contorno (PVC)

\par\bigskip}
{Prof: Felipe Figueiredo\par}
{\url{http://sites.google.com/site/proffelipefigueiredo}\par}
}

Versão: \verb|20150831|

%%%%%%%%%%%%%%%%%%%%%%%%


%%%%%%%%%%%%%%%%%%%%%%%%
\section{Objetivos de aprendizagem}

Ao final desta aula o aluno deve saber o que é e como resolver um
Problema de Valor Inicial (PVI) e um Problema de Valor de Contorno
(PVC).


\section{Pré-requitos da aula}

\begin{itemize}
\item Derivadas do seno e do cosseno
\item Valores do seno e do cosseno em $0$ e $\frac{\pi}{2}$
\end{itemize}

\section{Conteúdo}

O aluno deve consultar o livro texto nas seções 11.1 (PVI) e 11.10 (PVC)
para se aprofundar no conteúdo desta aula.

\subsection{Problema}

Um carro acelera em uma rua reta de acordo com a equação
$v'=\frac{1}{100}v$. Após $t=60$s, qual é a velocidade final do carro?

Para resolver esse problema, precisamos de uma informação adicional!

\subsection{Problema de Valor Inicial}

A família de soluções de uma EDO tem uma constante que não pode ser
determinada pelo método de resolução (que convencionamos chamar de
$K$). Para cada valor de $K\in \mathbb{R}$ temos uma função diferente,
e todas essas funções são soluções da EDO. Porém, cada função é
solução de um único problema. Assim, para resolver um problema,
precisamos encontrar o valor de $K$ que corresponde a ele.

Para isto, precisamos de uma informação adicional, chamada de {\em
  condição inicial} do problema. Vejamos alguns exemplos:

{\bf Primeiro exemplo:}

\begin{multicols}{2}
  Resolver o PVI:
    \begin{displaymath}
    \left\{
      \begin{array}{l}
        y'=y+5\\
        y(0)=1
      \end{array}
    \right.
  \end{displaymath}

{\bf Resolução:}

Vimos na aula passada como aplicar o método de separação de variáveis
nesta equação para encontrar a solução $y = Ke^t -5$. Só falta usar a
condição inicial $y(0) = 1$ para encontrar o valor de $K$.

Substituindo a condição inicial temos (para $t=0$):

$y(0) = Ke^{0}-5$

$1 = K - 5$

$K = 1+5$

$K=6$

Substituindo esse valor de $K$ na família de soluções, encontramos a
solução do PVI:

$y = 6e^t-5$
\end{multicols}

\newpage
\begin{multicols}{2}
{\bf Exemplo}

$
\left\{
  \begin{array}{l}
    2y'-2y +1 =0\\
    y(0)=\frac{5}{2}
  \end{array}
\right.
$

\bigskip
{\bf Resolução:}

$2y' = 2y - 1$

$y' = y - \frac{1}{2}$

\smallskip

$\frac{\ud y}{y - \frac{1}{2}} = \ud t$

\smallskip

$\ln \left(y - \frac{1}{2} \right) = t + c$

\smallskip

$y - \frac{1}{2}= K e^t$

\smallskip

$y(t) = K e^t + \frac{1}{2}$

\smallskip

Agora, substituindo a condição inicial:

\smallskip

$y(0) = Ke^0 +\frac{1}{2}$

\smallskip

$\frac{5}{2} = K + \frac{1}{2}$

\smallskip

$K=\frac{5}{2}-\frac{1}{2} = \frac{4}{2}$

\smallskip

$K=2$

Substituindo na solução, temos:

$y(t) = 2e^t +\frac{1}{2}$

%\columnbreak
\hrulefill

{\bf Exercício}

$
\left\{
  \begin{array}{l}
    y'=-5y\\
    y(0)=2
  \end{array}
\right.
$

\bigskip
{\bf Resolução:}

$\frac{\ud y}{y} = -5 \ud t$

$y(t) = Ke^{-5t}$

Substituindo o tempo inicial:

$y(0) = Ke^{-5 \times 0}$

$2 = K \times 1 \Rightarrow K= 2$

Substituindo $K$ na solução, temos:

$y(t) = 2e^{-5t}$
\end{multicols}

\hrulefill

\begin{multicols}{2}
{\bf Exemplo:} (EDO não linear)

$
\left\{
  \begin{array}{l}
    2yy'=t+2\\
    y(0)=2
  \end{array}
\right.
$

\bigskip
{\bf Resolução:}

$2\int y \ud y = \int (t+2) \ud t$

$2\frac{y^2}{2} = t^2 + 2t + c$

$y^2 = t^2 + 2t + c$

$y = \pm \sqrt{t^2 + 2t +K}$

Condição inicial:

$y(0)=\pm \sqrt{0^2 + 2\times 0 +K}$

$2 = \pm \sqrt{0 + 0 +K} = \pm \sqrt{K}$

Para encontrar $K$, basta elevar ambos os lados ao quadrado:

$2^2 = \left( \pm \sqrt{K} \right)^2$

$4 = K \Rightarrow K=4$

Substituindo $K$ para encontrar a solução:

$y(t) = \pm \sqrt{t^2 + 2t +4}$
\end{multicols}

\hrulefill

\subsection{Resolução do problema}

Vamos agora voltar no problema do carro, e considerar o seguinte PVI:

\begin{displaymath}
  \left\{
    \begin{array}{l}
      v'=\frac{1}{100}v\\
      \\
      v_0 = 20
    \end{array}
  \right.
\end{displaymath}

A família de soluções da EDO $v'=\frac{1}{100}v$ é

\begin{displaymath}
  v(t) = Ke^{(\frac{1}{100}t)}
\end{displaymath}

Substituindo o tempo inicial ($t=0$), temos:

\begin{displaymath}
  v(0) = Ke^{(\frac{1}{100}\times 0)}
\end{displaymath}
\begin{displaymath}
  20 = K \times 1
\end{displaymath}

Portanto a solução do PVI para um tempo $t$ qualquer:

\begin{displaymath}
  v(t) = 20e^{(\frac{1}{100}t)}
\end{displaymath}

Assim, para encontrarmos a velocidade após $t=60$s, basta substituir
esse valor na solução:

\begin{displaymath}
  v(60) = 20e^{(\frac{1}{100}\times 60)} \approx 36\frac{m}{s}
\end{displaymath}

Esta velocidade é $\approx 130\frac{km}{h}$.

Verifique em casa que se usarmos a velocidade inicial
$v_0=10\frac{m}{s}$ encontramos a velocidade terminal
$v(60) \approx 18\frac{m}{s} \approx 65\frac{km}{h}$

\subsection{Problema de Valor de Contorno}

Além de considerarmos o tempo inicial, podemos também resolver um
problema com tempo inicial e tempo final. Esse tipo de problema
aparece com EDOs de segunda ordem (derivada segunda).

Como aquecimento, vamos começar com um exercício de verificação de
solução de uma EDO de segunda ordem:

%\begin{multicols}{2}
{\bf Exercício:}

Verifique que $y(t) = \sin (t)$ é uma solução de $y''+y=0$.

\bigskip

{\bf Resolução}

%\bigskip

Derivando a função $y(t)$ temos:

$y' = \cos(t)$

$y'' = -\sin(t)$

Substituindo na EDO temos:

$\left(-\sin(t)\right) + \left(\sin(t)\right) =0$

$0=0$, portanto $y(t) = \sin(t)$ é uma solução da EDO.
%\end{multicols}

%\newpage

Agora vamos ver como se resolve um PVC, com as condições no tempo
inicial e no tempo final.

{\bf PVC}
\begin{displaymath}
  \left\{
    \begin{array}{l}
      y'' + y=0\\
      \smallskip
      y(0)=3\\
      \smallskip
      y(\frac{\pi}{2})=2
    \end{array}
  \right.
\end{displaymath}

Enquanto não aprendermos a resolver a EDO de segunda ordem, considere
a seguinte solução geral (verificar é um bom exercício para casa!):

\begin{displaymath}
  y(t) = K_1\sin(t) + K_2\cos(t)
\end{displaymath}

Observe que agora temos duas constantes $K_1$ e $K_2$, por isso
precisamos de duas condições de contorno.

Vamos começar substituindo o tempo inicial $t=0$:

\begin{displaymath}
  y(0)= K_1\cos(0) + K_2\sin(0)
\end{displaymath}
\begin{displaymath}
  3 = K_1 \times 1 + K_2 \times 0
\end{displaymath}
\begin{displaymath}
  3=K_1 \Rightarrow K_1=3
\end{displaymath}

Agora o tempo final $t=\frac{\pi}{2}$

\begin{displaymath}
  y(\frac{\pi}{2}) = K_1 \cos(\frac{\pi}{2}) + K_2\sin(\frac{\pi}{2})
\end{displaymath}
\begin{displaymath}
  2 = K_1 \times 0 + K_2\times 1
\end{displaymath}
\begin{displaymath}
  2 = K_2 \Rightarrow K_2=2
\end{displaymath}

Assim, substituindo as duas constantes $K_1$ e $K_2$ na função $y(t)$,
encontramos a solução do PVC:

\begin{displaymath}
  y(t) = 2\cos(t) + 3\sin(t)
\end{displaymath}

\end{document}
