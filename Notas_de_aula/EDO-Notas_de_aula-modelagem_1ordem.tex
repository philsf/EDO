\everymath{\displaystyle}
%\documentclass[pdftex,a4paper]{article}
\documentclass[a4paper]{article}
%%classes: article, report, book, proc, amsproc

%%%%%%%%%%%%%%%%%%%%%%%%
%% Misc
% para acertar os acentos
\usepackage[brazilian]{babel} 
%\usepackage[portuguese]{babel} 
% \usepackage[english]{babel}
% \usepackage[T1]{fontenc}
% \usepackage[latin1]{inputenc}
\usepackage[utf8]{inputenc}
\usepackage{indentfirst}
\usepackage{fullpage}
% \usepackage{graphicx} %See PDF section
\usepackage{multicol}
\setlength{\columnseprule}{0.5pt}
\setlength{\columnsep}{20pt}
%%%%%%%%%%%%%%%%%%%%%%%%
%%%%%%%%%%%%%%%%%%%%%%%%
%% PDF support

\usepackage[pdftex]{color,graphicx}
% %% Hyper-refs
\usepackage[pdftex]{hyperref} % for printing
% \usepackage[pdftex,bookmarks,colorlinks]{hyperref} % for screen

%% \newif\ifPDF
%% \ifx\pdfoutput\undefined\PDFfalse
%% \else\ifnum\pdfoutput > 0\PDFtrue
%%      \else\PDFfalse
%%      \fi
%% \fi

%% \ifPDF
%%   \usepackage[T1]{fontenc}
%%   \usepackage{aeguill}
%%   \usepackage[pdftex]{graphicx,color}
%%   \usepackage[pdftex]{hyperref}
%% \else
%%   \usepackage[T1]{fontenc}
%%   \usepackage[dvips]{graphicx}
%%   \usepackage[dvips]{hyperref}
%% \fi

%%%%%%%%%%%%%%%%%%%%%%%%


%%%%%%%%%%%%%%%%%%%%%%%%
%% Math
\usepackage{amsmath,amsfonts,amssymb}
% para usar R de Real do jeito que o povo gosta
\usepackage{amsfonts} % \mathbb
% para usar as letras frescas como L de Espaco das Transf Lineares
% \usepackage{mathrsfs} % \mathscr

% Oferecer seno e tangente em pt, com os comandos usuais.
\providecommand{\sin}{} \renewcommand{\sin}{\hspace{2pt}\mathrm{sen}}
\providecommand{\tan}{} \renewcommand{\tan}{\hspace{2pt}\mathrm{tg}}

% dt of integrals = \ud t
\newcommand{\ud}{\mathrm{\ d}}
%%%%%%%%%%%%%%%%%%%%%%%%



\begin{document}

%%%%%%%%%%%%%%%%%%%%%%%%
%% Título e cabeçalho
%\noindent\parbox[c]{.15\textwidth}{\includegraphics[width=.15\textwidth]{logo}}\hfill
\parbox[c]{.825\textwidth}{\raggedright%
  \sffamily {\LARGE

Equações Diferenciais: Notas de Aula

Modelagem matemática com EDOs de primeira ordem

\par\bigskip}
{Prof: Felipe Figueiredo\par}
{\url{http://sites.google.com/site/proffelipefigueiredo}\par}
}

Versão: \verb|20150905|

%%%%%%%%%%%%%%%%%%%%%%%%


%%%%%%%%%%%%%%%%%%%%%%%%
\section{Objetivos de aprendizagem}

Ao final desta aula o aluno deve saber \ldots


\section{Pré-requitos da aula}

\begin{itemize}
\item 
\item 
\end{itemize}

\section{Conteúdo}

O aluno deve consultar o livro texto na seção X.Y para se aprofundar
no conteúdo desta aula.


\subsection{Resistência do ar}


\subsection{Lei de Newton do Resfriamento (ou Aquecimento)}

Como exatamente uma tulipa de cerveja esquenta, ou uma xícara de café
esfria com o tempo?

Considere a temperatura $T(t)$ de um objeto, que está em um local onde
a temperatura ambiente $T_a$ é constante.

A {\bf Lei de Newton do Resfriamento} (ou do aquecimento) diz que
``{\em a temperatura do corpo tende a se igualar com a temperatura do
  ambiente, a uma taxa proporcional à diferença entre a diferença
  entre ambas}''.

Ora, a diferença entre a temperatura do objeto ($T$) e a temperatura
do ambiente ($T_a$) é simplesmente a subtração destes: $T - T_a$. A
variação da temperatura é sua derivada $T'$. Assim, a
proporcionalidade entre essas duas grandezas é dada pela equação

\begin{displaymath}
  T' = -k(T-T_a)
\end{displaymath}

Por que $k$ tem sinal negativo? Observe que conforme o tempo passa, a
diferença entre a temperatura do objeto e a temperatura do ambiente
vai diminuindo. A tendência é que, após um tempo muito grande, essas
temperaturas se igualem. Com isso, a temperatura $T$ do objeto deixa
de variar (ou seja, derivada $T'=0$).

Se você souber o valor da temperatura do ambiente, e qual é a
temperatura inicial do objeto, você pode montar e resolver um PVI
substituindo esses valores na equação acima.



\subsection{Meia vida: decaimento radioativo}


\end{document}
