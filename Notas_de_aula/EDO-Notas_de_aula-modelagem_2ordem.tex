%\everymath{\displaystyle}
%\documentclass[pdftex,a4paper]{article}
\documentclass[a4paper]{article}
%%classes: article, report, book, proc, amsproc

%%%%%%%%%%%%%%%%%%%%%%%%
%% Misc
% para acertar os acentos
\usepackage[brazilian]{babel} 
%\usepackage[portuguese]{babel} 
% \usepackage[english]{babel}
% \usepackage[T1]{fontenc}
% \usepackage[latin1]{inputenc}
\usepackage[utf8]{inputenc}
\usepackage{indentfirst}
\usepackage{fullpage}
% \usepackage{graphicx} %See PDF section
\usepackage{multicol}
\setlength{\columnseprule}{0.5pt}
\setlength{\columnsep}{20pt}
%%%%%%%%%%%%%%%%%%%%%%%%
%%%%%%%%%%%%%%%%%%%%%%%%
%% PDF support

\usepackage[pdftex]{color,graphicx}
% %% Hyper-refs
\usepackage[pdftex]{hyperref} % for printing
% \usepackage[pdftex,bookmarks,colorlinks]{hyperref} % for screen

%% \newif\ifPDF
%% \ifx\pdfoutput\undefined\PDFfalse
%% \else\ifnum\pdfoutput > 0\PDFtrue
%%      \else\PDFfalse
%%      \fi
%% \fi

%% \ifPDF
%%   \usepackage[T1]{fontenc}
%%   \usepackage{aeguill}
%%   \usepackage[pdftex]{graphicx,color}
%%   \usepackage[pdftex]{hyperref}
%% \else
%%   \usepackage[T1]{fontenc}
%%   \usepackage[dvips]{graphicx}
%%   \usepackage[dvips]{hyperref}
%% \fi

%%%%%%%%%%%%%%%%%%%%%%%%


%%%%%%%%%%%%%%%%%%%%%%%%
%% Math
\usepackage{amsmath,amsfonts,amssymb}
% para usar R de Real do jeito que o povo gosta
\usepackage{amsfonts} % \mathbb
% para usar as letras frescas como L de Espaco das Transf Lineares
% \usepackage{mathrsfs} % \mathscr

% Oferecer seno e tangente em pt, com os comandos usuais.
\providecommand{\sin}{} \renewcommand{\sin}{\hspace{2pt}\mathrm{sen}}
\providecommand{\tan}{} \renewcommand{\tan}{\hspace{2pt}\mathrm{tg}}

% dt of integrals = \ud t
\newcommand{\ud}{\mathrm{\ d}}
%%%%%%%%%%%%%%%%%%%%%%%%



\begin{document}

%%%%%%%%%%%%%%%%%%%%%%%%
%% Título e cabeçalho
%\noindent\parbox[c]{.15\textwidth}{\includegraphics[width=.15\textwidth]{logo}}\hfill
\parbox[c]{.825\textwidth}{\raggedright%
  \sffamily {\LARGE

Equações Diferenciais: Notas de Aula

Modelagem matemática com EDOs de segunda ordem

\par\bigskip}
{Prof: Felipe Figueiredo\par}
{\url{http://sites.google.com/site/proffelipefigueiredo}\par}
}

Versão: \verb|20151015|

%%%%%%%%%%%%%%%%%%%%%%%%


%%%%%%%%%%%%%%%%%%%%%%%%
\section{Objetivos de aprendizagem}

Ao final desta aula o aluno deve saber \ldots


\section{Pré-requitos da aula}

\begin{itemize}
\item 
\item 
\end{itemize}

\section{Conteúdo}

O aluno deve consultar o livro texto na seção X.Y para se aprofundar
no conteúdo desta aula.

\subsection{Problema}

Um tijolo de 2Kg está em repouso, pendurado em uma mola que tem
coeficiente de elasticidade 8$\frac{N}{cm}$. Você puxa o tijolo para
baixo, esticando a mola em 5cm e o solta. Assumindo que não há atrito,
o tijolo oscila para cima e para baixo em movimento harmônico
simples. Descrever este movimento no como a posição $s$ no espaço em
função do tempo, a amplitude, a frequência e o período da oscilação.

\subsection{A Lei de Hooke e o Oscilador Harmônico Simples}

Quando o tijolo está na posição $s$ força da mola que atua no tijolo é
dada pela lei de Hooke.

\begin{displaymath}
  F = -ks
\end{displaymath}

Sabemos pela segunda lei de Newton, que a resultante é $F=ma$ e que a
aceleração $a=s''$. Assim
\begin{displaymath}
  ma = -ks \Rightarrow ms'' +ks =0
\end{displaymath}
\begin{displaymath}
  s''+\frac{k}{m}s=0
\end{displaymath}

Que é uma EDO linear homogênea de segunda ordem com coeficientes
constantes. Para simplificar a notação, vamos substituir a constante
$\frac{k}{m}$ por $\omega^2$. Assim, $\omega=\sqrt{\frac{k}{m}}$. Após
essa transformação, a equação fica:

\begin{displaymath}
  s''+\omega^2s=0
\end{displaymath}

Essa mudança de variáveis tem algumas vantagens: (a) A equação fica
com uma única constante e (b) Esta constante tem uma interpretação
física: a frequência (angular) da oscilação e (c) $\omega$ já aparece
na solução da EDO!

\begin{displaymath}
  s(t) = K_1 \cos(\omega t) + K_2 \sin(\omega t)
\end{displaymath}

Podemos descobrir os valores de $K_1$ e $K_2$ tanto com valores
iniciais, ou valores de contorno. No problema inicial desta aula,
temos valores iniciais.

As características deste movimento são dados pelas seguintes fórmulas:

\begin{enumerate}
\item Frequência ($\omega$)
\begin{displaymath}
  \omega = \frac{k}{m}
\end{displaymath}
\item Período (T)
\begin{displaymath}
  T= \frac{2\pi}{\omega}
\end{displaymath}
\item Amplitude (A)
\begin{displaymath}
  A = \sqrt{K_1^2 + K_2^2}
\end{displaymath}
\end{enumerate}

Vamos agora usar estas fórmulas para descrever o movimento do tijolo:

Equação
\begin{displaymath}
  s''+4s=0
\end{displaymath}

Assim, podemos identificar que $\omega^2=\frac{8}{2}=4$, donde
$\omega=2$. O período é $T=\frac{2\pi}{2}=\pi$. Para encontrarmos a
amplitude, precisamos descobrir $K_1$ e $K_2$.

Como o tijolo foi deslocado para a posição inicial $s(0)=5$, e solto
com velocidade inicial $s'(0)=0$, podemos estas informações para
calcular  $K_1$ e $K_2$ e encontrar a solução do PVI.

\begin{displaymath}
  s(0) = K_1 \cos0 + K_2 \sin 0 = K_1 \Rightarrow K_1 = 5
\end{displaymath}

Calculando a derivada de $s(t)$ (exercício!) temos
$s'(t) = -2K_1 \sin t + 2K_2 \cos t$. Assim:

\begin{displaymath}
  s'(0) = -2K_1 \sin 0 + 2K_2 \cos 0 = 2K_2 \Rightarrow 2K_2 = 0
  \Rightarrow K_2=0
\end{displaymath}

Agora com os valores de $K_1$ e $K_2$ podemos escrever a solução do
PVI ($s(t) = 5\cos(2t)$), e calcular a amplitude do movimento:

\begin{displaymath}
  A = \sqrt{5^2 +0^2} = 5
\end{displaymath}

Conclusão: na ausência de atrito, o tijolo sobe e desce a 5cm da
posição de repouso.

Mas e se houver atrito?

\subsection{Oscilador Harmônico Amortecido}


\end{document}
