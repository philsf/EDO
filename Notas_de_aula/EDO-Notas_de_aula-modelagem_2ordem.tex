%\everymath{\displaystyle}
%\documentclass[pdftex,a4paper]{article}
\documentclass[a4paper]{article}
%%classes: article, report, book, proc, amsproc

%%%%%%%%%%%%%%%%%%%%%%%%
%% Misc
% para acertar os acentos
\usepackage[brazilian]{babel} 
%\usepackage[portuguese]{babel} 
% \usepackage[english]{babel}
% \usepackage[T1]{fontenc}
% \usepackage[latin1]{inputenc}
\usepackage[utf8]{inputenc}
\usepackage{indentfirst}
\usepackage{fullpage}
% \usepackage{graphicx} %See PDF section
\usepackage{multicol}
\setlength{\columnseprule}{0.5pt}
\setlength{\columnsep}{20pt}
%%%%%%%%%%%%%%%%%%%%%%%%
%%%%%%%%%%%%%%%%%%%%%%%%
%% PDF support

\usepackage[pdftex]{color,graphicx}
% %% Hyper-refs
\usepackage[pdftex]{hyperref} % for printing
% \usepackage[pdftex,bookmarks,colorlinks]{hyperref} % for screen

%% \newif\ifPDF
%% \ifx\pdfoutput\undefined\PDFfalse
%% \else\ifnum\pdfoutput > 0\PDFtrue
%%      \else\PDFfalse
%%      \fi
%% \fi

%% \ifPDF
%%   \usepackage[T1]{fontenc}
%%   \usepackage{aeguill}
%%   \usepackage[pdftex]{graphicx,color}
%%   \usepackage[pdftex]{hyperref}
%% \else
%%   \usepackage[T1]{fontenc}
%%   \usepackage[dvips]{graphicx}
%%   \usepackage[dvips]{hyperref}
%% \fi

%%%%%%%%%%%%%%%%%%%%%%%%


%%%%%%%%%%%%%%%%%%%%%%%%
%% Math
\usepackage{amsmath,amsfonts,amssymb}
% para usar R de Real do jeito que o povo gosta
\usepackage{amsfonts} % \mathbb
% para usar as letras frescas como L de Espaco das Transf Lineares
% \usepackage{mathrsfs} % \mathscr

% Oferecer seno e tangente em pt, com os comandos usuais.
\providecommand{\sin}{} \renewcommand{\sin}{\hspace{2pt}\mathrm{sen}}
\providecommand{\tan}{} \renewcommand{\tan}{\hspace{2pt}\mathrm{tg}}

% dt of integrals = \ud t
\newcommand{\ud}{\mathrm{\ d}}
%%%%%%%%%%%%%%%%%%%%%%%%



\begin{document}

%%%%%%%%%%%%%%%%%%%%%%%%
%% Título e cabeçalho
%\noindent\parbox[c]{.15\textwidth}{\includegraphics[width=.15\textwidth]{logo}}\hfill
\parbox[c]{.825\textwidth}{\raggedright%
  \sffamily {\LARGE

Equações Diferenciais: Notas de Aula

Modelagem matemática com EDOs de segunda ordem

\par\bigskip}
{Prof: Felipe Figueiredo\par}
{\url{http://sites.google.com/site/proffelipefigueiredo}\par}
}

Versão: \verb|20151015|

%%%%%%%%%%%%%%%%%%%%%%%%


%%%%%%%%%%%%%%%%%%%%%%%%
\section{Objetivos de aprendizagem}

Ao final desta aula o aluno deve conhecer os Osciladores Harmônicos
Simples e Amortecido.

% \section{Pré-requitos da aula}

% \begin{itemize}
% \item 
% \item 
% \end{itemize}

\section{Conteúdo}

O aluno deve consultar o livro texto nas seções 11.10 e 11.11 para se
aprofundar no conteúdo desta aula.

\subsection{Problema}

Um tijolo de 2Kg está em repouso, pendurado em uma mola que tem
coeficiente de elasticidade 8$\frac{N}{cm}$. Você puxa o tijolo para
baixo, esticando a mola em 5cm e o solta. Assumindo que não há atrito,
o tijolo oscila para cima e para baixo em movimento harmônico
simples. Descrever este movimento no como a posição $s$ no espaço em
função do tempo, a amplitude, a frequência e o período da oscilação.

\subsection{A Lei de Hooke e o Oscilador Harmônico Simples}

% Ao deslocar e soltar o tijolo, ele descreve um movimento oscilatório
% para cima e para baixo. Como não há atrito, ele faz esse vai-e-vem
% para sempre. Este é o Oscilador Harmônico Simples (OHS).

Quando o tijolo está na posição $s$ força da mola que atua no tijolo é
dada pela lei de Hooke.

\begin{displaymath}
  F = -ks
\end{displaymath}

Sabemos pela segunda lei de Newton, que a resultante é $F=m s''$, onde
$s''$ é aceleração.
\begin{displaymath}
  ms'' = -ks \Rightarrow ms'' +ks =0
\end{displaymath}
\begin{displaymath}
  s''+\frac{k}{m}s=0
\end{displaymath}


Para simplificar a notação, chamemos
$\omega=\sqrt{\frac{k}{m}}$. Assim $\omega^2 = \frac{k}{m}$, e a
equação do OHS fica apenas:

\begin{displaymath}
  s''+\omega^2s=0
\end{displaymath}

Essa mudança de variáveis tem algumas vantagens: (a) esta constante
tem uma interpretação física: a frequência (angular) da oscilação,
(b) $\omega$ já aparece na solução da EDO, facilitando a resolução e
(c) a equação fica mais simpática. As características deste movimento
são dados pelas seguintes fórmulas:

% \begin{displaymath}
%   s(t) = K_1 \cos(\omega t) + K_2 \sin(\omega t)
% \end{displaymath}

% Podemos descobrir os valores de $K_1$ e $K_2$ tanto com valores
% iniciais, ou valores de contorno. No problema inicial desta aula,
% temos valores iniciais.

\begin{itemize}
\item Frequência $\omega = \sqrt{\frac{k}{m}}$
\item Período $T= \frac{2\pi}{\omega}$
\item Amplitude $A = \sqrt{K_1^2 + K_2^2}$
\item Solução geral $s(t) = K_1 \cos (\omega t) + K_2 \sin (\omega t)$
\end{itemize}

{\bf Tijolo}

Vamos agora voltar ao nosso tijolo, e usar estas fórmulas para
descrever seu movimento:

\begin{displaymath}
  s''+4s=0
\end{displaymath}

Assim, podemos identificar que $\omega^2=\frac{8}{2}=4$, donde
$\omega=2$. A solução geral desta equação é $s(t) = K_1 \cos (2 t) + K_2 \sin (2 t)$. O período é $T=\frac{2\pi}{2}=\pi$. Para encontrarmos a
amplitude, precisamos descobrir $K_1$ e $K_2$.

Como o tijolo foi deslocado para a posição inicial $s(0)=5$, e solto
com velocidade inicial $s'(0)=0$, podemos estas informações para
calcular  $K_1$ e $K_2$ e encontrar a solução do PVI.

\begin{displaymath}
  s(0) = K_1 \cos0 + K_2 \sin 0 = K_1 \Rightarrow K_1 = 5
\end{displaymath}

Calculando a derivada de $s(t)$ (exercício!) temos
$s'(t) = -2K_1 \sin t + 2K_2 \cos t$. Assim:

\begin{displaymath}
  s'(0) = -2K_1 \sin 0 + 2K_2 \cos 0 = 2K_2 \Rightarrow 2K_2 = 0
  \Rightarrow K_2=0
\end{displaymath}

Agora com os valores de $K_1$ e $K_2$ podemos escrever a solução do
PVI ($s(t) = 5\cos(2t)$), e calcular a amplitude do movimento:

\begin{displaymath}
  A = \sqrt{5^2 +0^2} = 5
\end{displaymath}

Conclusão: na ausência de atrito, o tijolo se desloca para cima e para
baixo, atingindo sempre o deslocamento máximo de 5cm em relação à
posição de repouso.

Mas e se houver atrito?

\subsection{Oscilador Harmônico Amortecido}

Suponha agora que há atrito diminuindo a velocidade do tijolo. A
tendência é, portanto, que o tijolo diminua as oscilações até parar,
certo? Este é o Oscilador Harmônico Amortecido.

Digamos que a força de atrito é proporcional à velocidade $s'$ do
objeto (assim como no modelo da resistência do ar), ou seja, $R=as'$,
onde $a$ é o coeficiente de atrito.% e $s'$ é a velocidade.

A força resultante $F=ms''$ no objeto é a soma vetorial das forças
atuantes: a mola e o atrito.

% \begin{displaymath}
%   F = - as' -ks
% \end{displaymath}
\begin{displaymath}
  ms'' = -as' -ks \Rightarrow   ms'' +as' + ks =0
\end{displaymath}
% \begin{displaymath}
%   ms'' = -as' -ks
% \end{displaymath}
% \begin{displaymath}
%   ms'' +as' + ks =0
% \end{displaymath}
\begin{displaymath}
  s'' +\frac{a}{m}s' + \frac{k}{m}s  =0
\end{displaymath}

Por simplicidade, vamos chamar $b=\frac{a}{m}$ e
$c=\frac{k}{m}$. Assim, a equação acima fica $s''+bs'+cs=0$, cuja
equação característica é $r^2+br+c=0$.

Ao resolver esta última equação com o método da equação característica
($r^2+br+c=0$), podemos identificar quanto amortecimento há neste sistema.

\begin{enumerate}
\item Super-amortecido: não há oscilação
\item Sub-amortecido: o sistema oscila cada vez menos, até voltar para
  o equilíbrio
\end{enumerate}

Podemos distinguir entre estes dois tipos de amortecimento fazendo o
estudo das raízes da equação característica. Isto começa pelo
discriminante $\Delta$, para identificar se são reais (e quantas) ou
complexas.

Lembre-se que as soluções de uma EDO linear sempre envolvem funções
exponenciais. Estas funções podem crescer, ou decair. Como estamos
modelando um amortecimento, a amplitude das oscilações deve {\bf
  decair}, portanto precisamos obrigatoriamente de uma taxa negativa
nas exponenciais! Esta é a idéia que unifica todas as fórmulas abaixo:

\begin{enumerate}
\item Super-amortecido: $\Delta>0$, com ambas raízes $r_1<0$ e
  $r_2<0$.
\item Sub-amortecido: $\Delta<0$, com parte real $\alpha<0$.
\end{enumerate}

Para outra maneira fácil de distinguir os dois tipos de amortecimento
acima, basta lembrar que a oscilação vem das funções
trigonométricas. Assim, com $\Delta>0$ (raízes reais) não há
oscilação, apenas o decaimento (de volta ao repouso). Moleza.

Além destes dois tipos de amortecimento, podemos também identificar
outros dois interessantes:

\begin{itemize}
\item Criticamente amortecido (caso intermediário entre o super e o
  sub amortecido): $\Delta=0$, com raiz $r<0$. Para isto, basta que
  $b>0$ (pense: por quê?)
\item Não amortecido: $\Delta<0$ com parte real $\alpha=0$. Compare
  com o Oscilador Harmônico Simples.
\end{itemize}

\end{document}
